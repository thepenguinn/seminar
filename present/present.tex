\documentclass{beamer}

% % height = 5.9in width = 4.37in
% make this variable
%% \usepackage[paperheight = 4.37in, paperwidth = 5.9in, margin=0.2in]{geometry}
\usepackage[a4paper, margin=1in]{geometry}

\usepackage[export]{adjustbox}
\usepackage{graphicx}
\usepackage{xcolor}
\usepackage{circuitikz}
\usepackage{subfiles}
\usepackage{amsmath, amssymb}
\usepackage{enumitem}
\usepackage{nicematrix}
\usepackage{minted}
\usepackage{caption}
\usepackage{lmodern}
\usepackage{bookmark}
\usepackage{tabularx}
\usepackage{multirow}
\usepackage{multicol}
\usepackage{booktabs}
\usepackage{titlesec}
\usepackage{xspace}
\usepackage{varwidth}
\usepackage{titletoc}
\usepackage{epigraph}
\usepackage{etoolbox}
\usepackage{fontawesome5}
\usepackage[style = ieee]{biblatex} %Imports biblatex package
\usepackage{xfp}
\usepackage{xifthen}
\usepackage{tcolorbox}
\usepackage{xltabular}
\usepackage[T1]{fontenc}
\usepackage{setspace}
\usepackage[numbib]{tocbibind}

%% for bold textsc to work
\rmfamily % To load Latin Modern Roman and enable the following NFSS declarations.
% Declare that Latin Modern Roman (lmr) should take
% its bold (b) and bold extended (bx) weight, and small capital (sc) shape,
% from the corresponding Computer Modern Roman (cmr) font, for the T1 font encoding.
\DeclareFontShape{T1}{lmr}{b}{sc}{<->ssub*cmr/bx/sc}{}
\DeclareFontShape{T1}{lmr}{bx}{sc}{<->ssub*cmr/bx/sc}{}

\addbibresource{abstract.bib} %Import the bibliography file

\usepackage[bottom]{footmisc}

\usepackage{gensymb}
\usepackage{siunitx}
\usepackage{pgfplots}

\pgfplotsset{
    compat=newest,
    colormap={black}{rgb255=(0,0,0) rgb255=(0,0,0)}
}

\usetikzlibrary{intersections}
\usetikzlibrary{positioning}
\usetikzlibrary{calc}
\usetikzlibrary{ext.topaths.arcthrough}
\usetikzlibrary{decorations.markings}

\setlength{\arraycolsep}{0pt}
\renewcommand\arraystretch{1.5}

%% custom packages

\usepackage{colorscheme}
\usepackage{subtikzpicture}
\usepackage{generalcommands}

% alerts

\newcommand\alertCaution[1] {
    \begin{tcolorbox}[
            coltitle = colorAlertCaution,
            colbacktitle = colorAlertBgCaution,
            colback = colorAlertBgCaution,
            colframe = colorAlertBgCaution,
            title=Caution,
            fonttitle=\bfseries,
            detach title,
        ]
        \begin{minipage}[t]{0.18\textwidth}
            \begin{flushleft}
                \tcbtitle
            \end{flushleft}
        \end{minipage}
        \begin{minipage}[t]{0.8\textwidth}
            #1
        \end{minipage}
    \end{tcolorbox}
}

\newcommand\alertWarning[1] {
    \begin{tcolorbox}[
            coltitle = colorAlertWarning,
            colbacktitle = colorAlertBgWarning,
            colback = colorAlertBgWarning,
            colframe = colorAlertBgWarning,
            title=Warning,
            fonttitle=\bfseries,
            detach title,
        ]
        \begin{minipage}[t]{0.18\textwidth}
            \begin{flushleft}
                \tcbtitle
            \end{flushleft}
        \end{minipage}
        \begin{minipage}[t]{0.8\textwidth}
            #1
        \end{minipage}
    \end{tcolorbox}
}

\newcommand\alertImportant[1] {
    \begin{tcolorbox}[
            coltitle = colorAlertImportant,
            colbacktitle = colorAlertBgImportant,
            colback = colorAlertBgImportant,
            colframe = colorAlertBgImportant,
            title=Important,
            fonttitle=\bfseries,
            detach title,
        ]
        \begin{minipage}[t]{0.18\textwidth}
            \begin{flushleft}
                \tcbtitle
            \end{flushleft}
        \end{minipage}
        \begin{minipage}[t]{0.8\textwidth}
            #1
        \end{minipage}
    \end{tcolorbox}
}

\newcommand\alertTip[1] {
    \begin{tcolorbox}[
            coltitle = colorAlertTip,
            colbacktitle = colorAlertBgTip,
            colback = colorAlertBgTip,
            colframe = colorAlertBgTip,
            title=Tip,
            fonttitle=\bfseries,
            detach title,
        ]
        \begin{minipage}[t]{0.18\textwidth}
            \begin{flushleft}
                \tcbtitle
            \end{flushleft}
        \end{minipage}
        \begin{minipage}[t]{0.8\textwidth}
            #1
        \end{minipage}
    \end{tcolorbox}
}

\newcommand\alertNote[1] {
    \begin{tcolorbox}[
            coltitle = colorAlertNote,
            colbacktitle = colorAlertBgNote,
            colback = colorAlertBgNote,
            colframe = colorAlertBgNote,
            title=Note,
            fonttitle=\bfseries,
            detach title,
        ]
        \begin{minipage}[t]{0.18\textwidth}
            \begin{flushleft}
                \tcbtitle
            \end{flushleft}
        \end{minipage}
        \begin{minipage}[t]{0.8\textwidth}
            #1
        \end{minipage}
    \end{tcolorbox}
}

%% for paragraphs
\setlength{\parskip}{0.5\baselineskip}

%% \makeatletter
%% \def\maxwidth{\ifdim\Gin@nat@width>0.8\linewidth0.8\linewidth\else\Gin@nat@width\fi}
%% \def\maxheight{\ifdim\Gin@nat@height>0.9\textheight0.9\textheight\else\Gin@nat@height\fi}
%% \makeatother
%% % Scale images if necessary, so that they will not overflow the page
%% % margins by default, and it is still possible to overwrite the defaults
%% % using explicit options in \includegraphics[width, height, ...]{}
%% \setkeys{Gin}{width=\maxwidth,height=\maxheight,keepaspectratio}
%% % Set default figure placement to htbp

%% for fontawesome

%for scalling of fontawesome
\DeclareFontFamily{U}{fontawesome1}{}
\DeclareFontShape{U}{fontawesome1}{m}{n}{<->FontAwesome--fontawesomeone}{}
\DeclareFontFamily{U}{fontawesome2}{}
\DeclareFontShape{U}{fontawesome2}{m}{n}{<->FontAwesome--fontawesometwo}{}
\DeclareFontFamily{U}{fontawesome3}{}
\DeclareFontShape{U}{fontawesome3}{m}{n}{<->FontAwesome--fontawesomethree}{}
\DeclareFontFamily{U}{fontawesome5}{}
\DeclareFontShape{U}{fontawesome5}{m}{n}{<->FontAwesome--fontawesomefive}{}
\DeclareRobustCommand{\FAone}{\usefont{U}{fontawesome1}{m}{n}}
\DeclareRobustCommand{\FAtwo}{\usefont{U}{fontawesome2}{m}{n}}
\DeclareRobustCommand{\FAthree}{\usefont{U}{fontawesome3}{m}{n}}
\DeclareRobustCommand{\FAfive}{\usefont{U}{fontawesome5}{m}{n}}

\titleformat{\chapter} [display]
{\bfseries\normalfont\huge\filright\sffamily\vspace{-2cm}}
{\Large\textsc{chapter \num[minimum-integer-digits = 2]{\thechapter}} \vspace{1em}}
{1pc}
{\titlerule\vspace{0.5em}\scshape}
[\vspace{0.5em}{\titlerule[1pt]}]

%\titleformat{\section} [display]
%{\bfseries\normalfont\large\filright\sffamily}
%{}
%{2pt}
%{\scshape}
%{}

\setlength\epigraphwidth{9cm}
\setlength\epigraphrule{0pt}

\renewcommand{\epigraphflush}{center}

%% for graphicx
%% https://tex.stackexchange.com/questions/439918/set-default-value-for-max-width-of-includegraphics

%\expandafter\patchcmd\csname Gin@ii\endcsname
%{\setkeys {Gin}{#1}}
%{%
%    \setkeys {Gin}
%    {max width = 0.8\textwidth, max height = 0.4\textwidth, keepaspectratio, #1}%
%}
%{}{}

\def\IGXMaxWidth{\textwidth}
\def\IGXMaxHeight{\textheight}
\def\IGXDefaultOptionalArgs{keepaspectratio}

\makeatletter
\def\fps@figure{htbp}
\makeatother


% \usepackage{minted}

\usepackage[size=custom,width=16,height=9,scale=0.4]{beamerposter}
% \usepackage[size=custom, width=12.8, height=9.6, scale=0.4]{beamerposter}

\usepackage[T1]{fontenc}
\usepackage{tcolorbox}
\usepackage{graphicx}
\usepackage{lmodern}
\usepackage{tabularx}
\usepackage{tikz}
\usepackage{blindtext}
\usepackage{svg}
\usepackage{fontspec}
\usepackage{booktabs}
\usepackage[style = ieee]{biblatex}
\usepackage{blindtext}
\usepackage{setspace}
\usepackage{minted}
\usepackage{nicematrix}

\usetikzlibrary{calc}
\usetikzlibrary{intersections,decorations.markings}

\usemintedstyle{dracula}

\makeatletter
\tikzset{
  use path for main/.code={%
    \tikz@addmode{%
      \expandafter\pgfsyssoftpath@setcurrentpath\csname tikz@intersect@path@name@#1\endcsname
    }%
  },
  use path for actions/.code={%
    \expandafter\def\expandafter\tikz@preactions\expandafter{\tikz@preactions\expandafter\let\expandafter\tikz@actions@path\csname tikz@intersect@path@name@#1\endcsname}%
  },
  use path/.style={%
    use path for main=#1,
    use path for actions=#1,
  }
}
\setsansfont{Garamond.ttf}

% \usefonttheme{serif}
% \usepackage{beamercolorthemedracula}

% \addbibresource{abstract.bib} %Import the bibliography file

\usetheme{metropolis}

\definecolor{forestbg}      {RGB} {31, 30, 24}

\title{TinyML in Agriculture Sector}

\subtitle{Awakening of Minds in Low Power Edge Devices}

\date{\today}

\author{
    Daniel V Mathew
}

\institute{Rajiv Gandhi Institute of Technology, Kottayam}

\def\supervisor{
    Prof. Sujithamol S
}

\def\leafColorPrimaryName{Sage}
\def\leafColorPrimaryRGB{138, 184, 114}
\def\leafColorPrimaryHex{\#8AB872}
\definecolor{leafColorPrimary} {RGB} {\leafColorPrimaryRGB}
%%
\def\leafColorSecondaryName{Cactus Spike}
\def\leafColorSecondaryRGB{204, 224, 153}
\def\leafColorSecondaryHex{\#CCE099}
\definecolor{leafColorSecondary} {RGB} {\leafColorSecondaryRGB}

\definecolor{colorBlue}   {HTML} {33539E}
\definecolor{colorSky}    {HTML} {7FACD6}
\definecolor{colorViolet} {HTML} {BFB8DA}
\definecolor{colorPink}   {HTML} {E8B7D4}
\definecolor{colorMaroon} {HTML} {A5678E}

\definecolor{colorRampart} {HTML} {BDB7B7}
\definecolor{colorCaramel} {HTML} {B89B93}
\definecolor{colorCortex} {HTML} {A59691}
\definecolor{colorDove} {HTML} {B6AFA5}
\definecolor{colorCathedral} {HTML} {ABA7A6}
\definecolor{colorWind} {HTML} {CAC4C4}

\definecolor{colorRed} {HTML} {E44E54}
\definecolor{colorLightRed} {HTML} {F46367}
\definecolor{colorVeryLightRed} {HTML} {D1AE97}
\definecolor{colorDarkRed} {HTML} {581319}
\definecolor{colorDarkGreen} {HTML} {2E4330}
\definecolor{colorLightGreen} {HTML} {C0D2A3}

\definecolor{colorLightBrown} {HTML} {A8867B}
\definecolor{colorDarkBrown} {HTML} {9EC2AD}

\definecolor{colorBrown} {HTML} {9F774E}

\definecolor{colorTest} {HTML} {133330}

\renewcommand\alert[1] {%
    {\color{colorRed}#1}%
}

\newcommand\tableBox[2] {
    \begin{tcolorbox}[
            coltitle = leafColorPrimary,
            colbacktitle = leafColorSecondary,
            colback = leafColorSecondary,
            colframe = leafColorSecondary,
            title=#1,
            fonttitle=\bfseries,
            detach title,
        ]
        \begin{minipage}[t]{0.1\textwidth}
            \begin{flushleft}
                \tcbtitle
            \end{flushleft}
        \end{minipage}
        \begin{minipage}[t]{0.9\textwidth}
            #2
        \end{minipage}
    \end{tcolorbox}
}

% \setbeamertemplate{background} {
%     \begin{tikzpicture} [remember picture, overlay]
%         \filldraw [
%             colorWind,
%         ]
%         (current page.north west) rectangle (current page.south east)
%         ;
%     \end{tikzpicture}
% }

% \setbeamertemplate{frametitle}{
%     \begin{tikzpicture} [remember picture, overlay]
%         \node (title) at (current page.north west) [
%             black, anchor = north west
%         ] {\strut\insertframetitle\strut};
%         \draw[
%             black,
%             very thick,
%             line cap = round,
%         ]
%         (title.east) -- (title.east -| current page.east)
%         ;
%     \end{tikzpicture}%
% }

\setbeamercolor{frametitle}{%
  fg=white,
  bg=colorMaroon
}

\newcommand\tikzframetitleleft[2] {
    \draw (current page.north west) ++(0.75cm, -0.75cm) coordinate (frametitle) ;
    \node [anchor = north west] (frametitle) at (frametitle) {
        \begin{minipage} {8cm}
            \setstretch{0.75}
            \huge
            {\Large #1} \\ #2
        \end{minipage}
    }
    ;
}

\newcommand\tikzframetitleright[2] {
    \draw (current page.north east) ++(-0.75cm, -0.75cm) coordinate (frametitle) ;
    \node [anchor = north east] (frametitle) at (frametitle) {
        \begin{minipage} {8cm}
            \raggedleft
            \setstretch{0.75}
            \huge
            {\Large #1} \\ #2
        \end{minipage}
    }
    ;
}

\colorlet{fgcol}{colorDarkGreen}

\def\markMainInnerRadius{0.175cm}
\def\markMainOuterRadius{0.3cm}
\def\markSubRadius{0.075cm}
\def\markCornerRadius{0.1cm}

\begin{document}

\maketitle

% \begin{frame} {Introduction}
%     \begin{tikzpicture}
%
%         \draw (0, 0) coordinate (tmp);
%         \foreach \x in {
%             colorDarkGreen,
%             colorRed,
%             colorBlue,
%             colorSky,
%             colorViolet,
%             colorPink,
%             colorMaroon,
%             %%
%             colorRampart,
%             colorCaramel,
%             colorCortex,
%             colorDove,
%             colorCathedral%
%         } {
%             \filldraw [ \x ] (tmp) circle (1cm) (tmp) ++(1.5, 0) coordinate (tmp);
%         }
%
%     \end{tikzpicture}
% \end{frame}

%% Making Sense of Data
\begin{frame} {}

    \begin{tikzpicture} [
            remember picture,
            overlay,
        ]

        \draw (0, 0)
        coordinate (origin)
        (current page.east) ++(1, 0) coordinate (edge)
        ;

        \begin{scope}
            %% Humidity
            \draw []
            (current page.south) ++(1.0cm, 2.5cm) coordinate (humidityMarkCenter)
            ;
            \path [name path = markHumeHandle, rounded corners = \markCornerRadius]
            (humidityMarkCenter) ++(135:\markMainOuterRadius) coordinate (humidityMarkFirst)
            (humidityMarkCenter) ++(135:(0.75cm + \markMainOuterRadius) coordinate (humidityMarkSecond)
            (humidityMarkSecond) ++(-1, 0) coordinate (humidityMarkThird)
            (humidityMarkFirst) -- (humidityMarkSecond) -- (humidityMarkThird)
            ;
            \draw [use path = markHumeHandle, very thick] ;
            ;
            \filldraw [fgcol]
            (humidityMarkThird) circle (\markSubRadius)
            ;
            \draw [fgcol]
            (humidityMarkThird) node [left = 6pt] {
                Humidity Data
            }
            ;

            %% Temperature
            \draw []
            (current page.south) ++(5.0cm, 6.55cm) coordinate (temperatureMarkCenter)
            ;
            \path [name path = markTempHandle, rounded corners = \markCornerRadius]
            (temperatureMarkCenter) ++(135:\markMainOuterRadius) coordinate (temperatureMarkFirst)
            (temperatureMarkCenter) ++(135:(0.75cm + \markMainOuterRadius) coordinate (temperatureMarkSecond)
            (temperatureMarkSecond) ++(-1, 0) coordinate (temperatureMarkThird)
            (temperatureMarkFirst) -- (temperatureMarkSecond) -- (temperatureMarkThird)
            ;
            \draw [use path = markTempHandle, very thick] ;
            ;
            \filldraw [fgcol]
            (temperatureMarkThird) circle (\markSubRadius)
            ;
            \draw [fgcol]
            (temperatureMarkThird) node [left = 6pt] {
                Temperature Data
            }
            ;

            \clip [rotate = 45, rounded corners = 0.7cm]
            (edge) ++(-10, -15) rectangle ++(20, 20)
            ;

            \begin{scope}
                \node [anchor = east] (img) at (edge) {
                    \includegraphics[height = \paperheight]{low.jpg}
                }
                ;

                %% Humidity
                \filldraw [white!90!fgcol]
                (humidityMarkCenter) circle (\markMainInnerRadius)
                ;
                \draw [white!90!fgcol, very thick]
                (humidityMarkCenter) circle (\markMainOuterRadius)
                ;
                \draw [use path = markHumeHandle, white!90!fgcol, very thick] ;

                %% Temperature
                \filldraw [white!90!fgcol]
                (temperatureMarkCenter) circle (\markMainInnerRadius)
                ;
                \draw [white!90!fgcol, very thick]
                (temperatureMarkCenter) circle (\markMainOuterRadius)
                ;
                \draw [use path = markTempHandle, white!90!fgcol, very thick] ;

            \end{scope}
        \end{scope}

        \tikzframetitleleft{Making} {Sense of Data}

        \draw (frametitle.south west) ++(0, -0.5cm) coordinate (content) ;
        \node [anchor = north west]  (content) at (content) {
            \begin{minipage} {8.75cm}

                Every system is in some sense a \alert{Data Processing System}.
                They \alert{Convert / Extract / Manipulate} Data in different ways.
                There are several kinds of data, such as \alert{Visual}, \alert{Auditory},
                \alert{Sensory}, etc. Goal of almost any system is to make the most use of
                the data available to it.

            \end{minipage}
        }
        ;

        \draw (0, 0)
        coordinate (origin)
        (current page.west) coordinate (edge)
        ;
        \begin{scope} [transform canvas = {yshift = -2.2cm, xshift = 0.5cm}]
            \draw []
            (edge) ++(2cm, -0.5cm) coordinate (pixelMarkCenter)
            ;
            \path [name path = markHandle, rounded corners = \markCornerRadius]
            (pixelMarkCenter) ++(-45:\markMainOuterRadius) coordinate (pixelMarkFirst)
            (pixelMarkCenter) ++(-45:(1cm + \markMainOuterRadius) coordinate (pixelMarkSecond)
            (pixelMarkSecond) ++(1, 0) coordinate (pixelMarkThird)
            (pixelMarkFirst) -- (pixelMarkSecond) -- (pixelMarkThird)
            ;
            \draw [use path = markHandle, very thick] ;
            \filldraw [fgcol]
            (pixelMarkThird) circle (\markSubRadius)
            ;
            \draw [fgcol]
            (pixelMarkThird) node [right = 6pt] {
                Pixel Data
            }
            ;
            \clip [rotate = -45, rounded corners = 0.7cm]
            (edge) rectangle ++(2.5, 2.5)
            ;
            \begin{scope}
                \node [anchor = west] (pixel) at (edge) {
                    \includegraphics[height = \paperheight]{low.jpg}
                }
                ;
                \draw [thin, step = 0.1, white, opacity = 0.40]
                (edge) ++(-1, -2) grid ++(5, 5)
                ;
                \filldraw [white!90!fgcol]
                (pixelMarkCenter) circle (\markMainInnerRadius)
                ;

                \draw [white!90!fgcol, very thick]
                (pixelMarkCenter) circle (\markMainOuterRadius)
                ;
                \draw [use path = markHandle, white!90!fgcol, very thick] ;
            \end{scope}
        \end{scope}

    \end{tikzpicture}

\end{frame}

%% Learning to Ask the Right Questions
\begin{frame} {}

    \begin{tikzpicture} [
            remember picture,
            overlay,
        ]
        \tikzframetitleleft{Asking}{The Right Questions}

        \draw (frametitle.south west) ++(0, -0.5cm) coordinate (content) ;
        \node [anchor = north west]  (content) at (content) {
            \begin{minipage} {0.45\paperwidth}
                In order to make sense of the data, we need to ask the right questions about the data.
                The sequence of these questions is known as an \alert{Algorithm}.
            \end{minipage}
        }
        ;

        \filldraw [colorTest]
        (current page.north east) ++(-7,0) coordinate (workFlowTop)
        (workFlowTop) rectangle (current page.south east)
        ;

        \draw (workFlowTop) ++(0, -0.2) coordinate (tmp);
        \foreach \x in {1, 2, 3, 4} {
            \draw [very thick, white, rounded corners = 0.2cm, line cap = round]
            (tmp) ++(0, -1) -- ++(3.5, -0.8) -- ++(3.5, 0.8)
            (tmp) ++(0, -0.25\paperheight) coordinate (tmp)
            ;
        }

        \draw ($(workFlowTop)!0.5!(current page.north east)$) ++(0, -0.8) coordinate (tmp);
        \foreach \x in {
            Finding a Problem,
            Selecting an Algorithm,
            Implementing the Algorithm,
            Solving the Problem%
        } {
            \draw [ultra thick, white, rounded corners = 0.2cm]
            (tmp)
            node [anchor = center] {
                \begin{minipage} {7cm}
                    \centering
                    \x
                \end{minipage}
            }
            % (tmp) ++(0, -1) -- ++(3.5, -1.25) -- ++(3.5, 1.25)
            (tmp) ++(0, -0.25\paperheight) coordinate (tmp)
            ;
        }

        %% Methodology

        \begin{scope}
            \draw
            (current page.center) ++(1.75cm, 0) coordinate (workFlowMarkCenter)
            % (current page.center) ++(1.6cm, 0.2cm) coordinate (workFlowMarkCenter)
            ;
            \path [name path = markWorkHandle, rounded corners = \markCornerRadius]
            (workFlowMarkCenter) ++(-180:\markMainOuterRadius) coordinate (workFlowMarkFirst)
            (workFlowMarkFirst) ++(-1, 0) coordinate (workFlowMarkSecond)
            (workFlowMarkSecond) ++(-135:0.75cm) coordinate (workFlowMarkThird)
            (workFlowMarkFirst) -- (workFlowMarkSecond) -- (workFlowMarkThird)
            ;
            \draw [use path = markWorkHandle, very thick] ;
            \filldraw [fgcol]
            (workFlowMarkThird) circle (\markSubRadius)
            ;
            \draw [fgcol]
            (workFlowMarkThird) node (workFlowMark) [left = 6pt] {
                Typical Way of Solving a Problem
            }
            ;
            \clip (workFlowTop) rectangle (current page.south east);
            \begin{scope}
                \filldraw [white!90!fgcol]
                (workFlowMarkCenter) circle (\markMainInnerRadius)
                ;
                \draw [white!90!fgcol, very thick]
                (workFlowMarkCenter) circle (\markMainOuterRadius)
                ;
                \draw [white!90!fgcol, use path = markWorkHandle, very thick] ;
            \end{scope}
        \end{scope}

        \draw (content.south west |- workFlowMark.south west) ++(0, -0.5cm) coordinate (content) ;
        \node [anchor = north west]  (content) at (content) {
            \begin{minipage} {0.45\paperwidth}
                But what if the sheer number of questions one must ask to get to the right answer gets
                so overwhelming?
            \end{minipage}
        }
        ;

        \draw (content.south west) ++(0, -0.5cm) coordinate (content) ;
        \node [anchor = north west]  (content) at (content) {
            \begin{minipage} {0.45\paperwidth}
                But before that...
            \end{minipage}
        }
        ;

    \end{tikzpicture}

\end{frame}

%% Ways of Asking Questions
\begin{frame}[fragile] {}

    \begin{tikzpicture} [
            remember picture,
            overlay
        ]
        \tikzframetitleright {Ways of} {Asking Questions}

        \begin{scope} [transform canvas = {xshift = 0.2cm}]
            \begin{scope}
                \draw
                (current page.north) ++(-0.75, -0.75) coordinate (ifelseMarkCenter)
                ;
                \path [name path = ifelseHandle, rounded corners = \markCornerRadius]
                (ifelseMarkCenter) ++(-45:\markMainOuterRadius) coordinate (ifelseMarkFirst)
                (ifelseMarkCenter) ++(-45:(2.85cm + \markMainOuterRadius) coordinate (ifelseMarkSecond)
                (ifelseMarkSecond) ++(1.4, 0) coordinate (ifelseMarkThird)
                (ifelseMarkFirst) -- (ifelseMarkSecond) -- (ifelseMarkThird)
                ;
                \draw [use path = ifelseHandle, very thick] ;
                \filldraw [
                    colorLightGreen,
                    rounded corners = 0.7cm,
                ]
                (current page.north west) ++(0, 0.1cm) ++(0,2) coordinate (tmpStart)
                (current page.center) ++(0, 0.1cm) coordinate (tmpEnd)
                (tmpStart) rectangle (tmpEnd)
                ;
                \filldraw [fgcol]
                (ifelseMarkThird) circle (\markSubRadius)
                ;
                \draw [fgcol]
                (ifelseMarkThird) node (ifelse) [right = 6pt] {
                    Using If-Else Statements
                }
                ;
                \clip [
                    rounded corners = 0.7cm,
                ]
                (current page.north west) ++(0, 0.1cm) ++(0,2) coordinate (tmpStart)
                (current page.center) ++(0, 0.1cm) coordinate (tmpEnd)
                (tmpStart) rectangle (tmpEnd)
                ;
                \begin{scope}
                    \filldraw [white!90!fgcol]
                    (ifelseMarkCenter) circle (\markMainInnerRadius)
                    ;
                    \draw [white!90!fgcol, very thick]
                    (ifelseMarkCenter) circle (\markMainOuterRadius)
                    ;
                    \draw [white!90!fgcol, use path = ifelseHandle, very thick] ;
                \end{scope}
            \end{scope}

            % \draw [thin, step = 0.1, white, opacity = 0.40]
            % (current page.north west) ++(-2,2) grid (current page.center)
            % ;

            \begin{scope}
                \draw
                (current page.south) ++(-0.75, 1.0) coordinate (matrixMarkCenter)
                ;
                \path [name path = matrixHandle, rounded corners = \markCornerRadius]
                (matrixMarkCenter) ++(45:\markMainOuterRadius) coordinate (matrixMarkFirst)
                (matrixMarkCenter) ++(45:(2.0cm + \markMainOuterRadius) coordinate (matrixMarkSecond)
                (matrixMarkSecond) ++(1.35, 0) coordinate (matrixMarkThird)
                (matrixMarkFirst) -- (matrixMarkSecond) -- (matrixMarkThird)
                ;
                \draw [use path = matrixHandle, very thick] ;
                \filldraw [
                    colorDarkBrown,
                    rounded corners = 0.7cm,
                ]
                (current page.south west) ++(0, -0.1cm) ++(0,-2) coordinate (tmpStart)
                (current page.center) ++(0, -0.1cm) coordinate (tmpEnd)
                (tmpStart) rectangle (tmpEnd)
                ;
                \filldraw [fgcol]
                (matrixMarkThird) circle (\markSubRadius)
                ;
                \draw [fgcol]
                (matrixMarkThird) node (matrix) [right = 6pt] {
                    Using Matrix Transformation
                }
                ;
                \clip [
                    rounded corners = 0.7cm,
                ]
                (current page.south west) ++(0, -0.1cm) ++(0,-2) coordinate (tmpStart)
                (current page.center) ++(0, -0.1cm) coordinate (tmpEnd)
                (tmpStart) rectangle (tmpEnd)
                ;
                \begin{scope}
                    \filldraw [white!90!fgcol]
                    (matrixMarkCenter) circle (\markMainInnerRadius)
                    ;
                    \draw [white!90!fgcol, very thick]
                    (matrixMarkCenter) circle (\markMainOuterRadius)
                    ;
                    \draw [white!90!fgcol, use path = matrixHandle, very thick] ;
                \end{scope}
            \end{scope}

            % \draw [thin, step = 0.1, white, opacity = 0.40]
            % (current page.south west) ++(-2,-2) grid (current page.center)
            % ;

            \draw (current page.north west) ++(0.5cm, -0.5cm) coordinate (content) ;
            \node [anchor = north west]  (content) at (content) {
                \begin{minipage} {0.55\paperwidth}
                    \begin{minted}[breaklines, autogobble, mathescape] {text}
                        if this_pixel == this_color:
                            if this_pixel is at_edge:
                                # more and more if statements
                        elif this_pixel == that_color:
                            if this_pixel is in_the_middle:
                                # many more
                    \end{minted}
                \end{minipage}
            }
            ;

            \draw (current page.south west) ++(0.5cm, 0.5cm) coordinate (content) ;
            \node [anchor = south west]  (content) at (content) {
                \begin{minipage} {0.425\paperwidth}
                    \centering
                    $\begin{pNiceMatrix}
                        c_{11} & c_{12} & c_{13} & c_{14} & \cdots & c_{1n} \\
                        c_{21} & c_{22} & c_{23} & c_{24} & \cdots & c_{2n} \\
                        c_{31} & c_{32} & c_{33} & c_{34} & \cdots & c_{3n} \\
                        c_{41} & c_{42} & c_{43} & c_{44} & \cdots & c_{4n} \\
                        \vdots & \vdots & \vdots & \vdots & \ddots & \vdots\\
                        c_{m1} & c_{m2} & c_{m3} & c_{m4} & \cdots & c_{mn} \\
                    \end{pNiceMatrix}$
                \end{minipage}
            }
            ;

        \end{scope}

        \draw (ifelse.south east) ++(0.2cm, -0.5cm) coordinate (content) ;
        \node [anchor = north east]  (content) at (content) {
            \begin{minipage} {6.5cm}
                There are several ways to ask the same question. One might be able to
                to write an Algorithm for parsing the pixel data just by using \alert{If-Else}
                statements.
            \end{minipage}
        }
        ;

        \draw (matrix.south east) ++(0.2cm, -0.5cm) coordinate (content) ;
        \node [anchor = north east]  (content) at (content) {
            \begin{minipage} {6.5cm}
                Or one can directly transform input into output using a \alert{Matrix Tranformation}.
            \end{minipage}
        }
        ;

    \end{tikzpicture}

\end{frame}

%% Guess Who's Good at Crunching Matrices?
\begin{frame} {}

    \begin{tikzpicture} [
            remember picture,
            overlay
        ]

        \tikzframetitleright{Guess Who's} {Good at Crunching Matrices?}

        \draw (0, 0)
        coordinate (origin)
        (current page.south west) coordinate (edge)
        ;

        \begin{scope} [
                transform canvas = {
                    yshift = -13,
                    xshift = -13,
                }
            ]
            \begin{scope} [
                    transform canvas = {
                        yshift = 0.1cm,
                        xshift = 0.1cm,
                    }
                ]
                \draw
                (current page.north) ++(-2.0, -1.6) coordinate (cpuMarkCenter)
                ;
                \path [name path = cpuHandle, rounded corners = \markCornerRadius]
                (cpuMarkCenter) ++(0:\markMainOuterRadius) coordinate (cpuMarkFirst)
                (cpuMarkCenter) ++(0:(1.5cm + \markMainOuterRadius) coordinate (cpuMarkSecond)
                (cpuMarkSecond) ++(-45:2.1cm) coordinate (cpuMarkThird)
                (cpuMarkThird) ++(5, 0) coordinate (cpuMarkFourth)
                (cpuMarkFirst) -- (cpuMarkSecond) -- (cpuMarkThird) -- (cpuMarkFourth)
                ;
                \draw [use path = cpuHandle, very thick] ;
                \filldraw [fgcol]
                (cpuMarkFourth) circle (\markSubRadius)
                ;
                \draw [fgcol]
                (cpuMarkFourth) node (cpu) [right = 6pt] {
                    CPUs
                }
                ;
                \begin{scope}
                    \clip [rotate = 45, rounded corners = 0.7cm]
                    (edge) ++(0.25, 0.25) rectangle ++(10, 10)
                    ;
                    \draw
                    (edge) ++(3.00, 0.35) coordinate (tmp)
                    ;
                    \node [anchor = south] (img) at (tmp) {
                        \includegraphics[height = \paperheight]{pic/cpu_hald-clut.jpg}
                    }
                    ;
                    \filldraw [white!90!fgcol]
                    (cpuMarkCenter) circle (\markMainInnerRadius)
                    ;
                    \draw [white!90!fgcol, very thick]
                    (cpuMarkCenter) circle (\markMainOuterRadius)
                    ;
                    \draw [white!90!fgcol, use path = cpuHandle, very thick] ;
                \end{scope}

                \draw (cpu.south east) ++(0, -0.3cm) coordinate (content) ;
                \node [anchor = north east]  (content) at (content) {
                    \begin{minipage} {6.0cm}
                        \raggedleft
                        \alert{CPUs} are much more capable at processing matrices that anyone of us, humans.
                        % \alert{Computers}, especially \alert{GPUs} can crunch through matrices.
                    \end{minipage}
                }
                ;
            \end{scope}

            \begin{scope} [
                    transform canvas = {
                        yshift = -0.5cm,
                        xshift = -0.5cm,
                    }
                ]
                \draw
                (current page.south) ++(1.75, 3.25) coordinate (gpuMarkCenter)
                ;
                \path [name path = gpuHandle, rounded corners = \markCornerRadius]
                (gpuMarkCenter) ++(45:\markMainOuterRadius) coordinate (gpuMarkFirst)
                (gpuMarkCenter) ++(45:(1.0cm + \markMainOuterRadius) coordinate (gpuMarkSecond)
                (gpuMarkSecond) ++(4.25, 0) coordinate (gpuMarkThird)
                (gpuMarkFirst) -- (gpuMarkSecond) -- (gpuMarkThird)
                ;
                \draw [use path = gpuHandle, very thick] ;
                \filldraw [fgcol]
                (gpuMarkThird) circle (\markSubRadius)
                ;
                \draw [fgcol]
                (gpuMarkThird) node (gpu) [right = 6pt] {
                    GPUs
                }
                ;
                \begin{scope}
                    \clip [rotate = 45, rounded corners = 0.7cm]
                    (edge) ++(0.25, -0.25) rectangle ++(10, -10)
                    ;
                    \draw
                    (edge) ++(0, -0.5) coordinate (tmp)
                    ;
                    \node [anchor = south west] (img) at (tmp) {
                        \includegraphics[height = 0.9\paperheight]{pic/gpu_hald-clut.jpg}
                    }
                    ;
                    \filldraw [white!90!fgcol]
                    (gpuMarkCenter) circle (\markMainInnerRadius)
                    ;
                    \draw [white!90!fgcol, very thick]
                    (gpuMarkCenter) circle (\markMainOuterRadius)
                    ;
                    \draw [white!90!fgcol, use path = gpuHandle, very thick] ;
                \end{scope}
                \draw (gpu.south east) ++(0, -0.3cm) coordinate (content) ;
                \node [anchor = north east]  (content) at (content) {
                    \begin{minipage} {4.0cm}
                        \raggedleft \alert{GPUs} can exploit parallelizable nature of matrix operations.
                        % \alert{Computers}, especially \alert{GPUs} can crunch through matrices.
                    \end{minipage}
                }
                ;
            \end{scope}
        \end{scope}

    \end{tikzpicture}


\end{frame}

%% Learning to Ask Questions
\begin{frame} {}

    \begin{tikzpicture} [
            remember picture,
            overlay
        ]

        \tikzframetitleleft{Learning to} {Ask Questions}

        \draw (0, 0)
        coordinate (origin)
        (current page.north east) coordinate (edge)
        ;

        \begin{scope} [
                transform canvas = {
                    yshift = -0.5cm,
                    xshift = -1.25cm,
                }
            ]
            \begin{scope}
                \clip [rotate = -45, rounded corners = 0.7cm]
                (edge) ++(7, 6) rectangle ++(-12, -12)
                ;
                \draw
                (edge) ++(-2, 0.625) coordinate (tmp)
                ;
                \node [anchor = north] (img) at (tmp) {
                    \includegraphics[height = \paperheight]{pic/snetwork_hald-clut.jpg}
                }
                ;
            \end{scope}

        \end{scope}

        \draw (frametitle.south west) ++(0, -0.3cm) coordinate (content) ;
        \node [anchor = north west]  (content) at (content) {
            \begin{minipage} {6.8cm}
                We haven't addressed how we are going to know the right questions to ask about
                a data yet.
            \end{minipage}
        }
        ;

        \draw (content.south west) ++(0, -0.3cm) coordinate (content) ;
        \node [anchor = north west]  (content) at (content) {
            \begin{minipage} {8.1cm}
                The \alert{coefficients / weights} in our question matrix decides the output
                we get.
            \end{minipage}
        }
        ;

        \draw (content.south west) ++(0, -0.3cm) coordinate (content) ;
        \node [anchor = north west]  (content) at (content) {
            \begin{minipage} {9.3cm}
                And the game of finding these coefficients / weights is famously known under
                the umbrella term \alert{Machine Learning}.
            \end{minipage}
        }
        ;

    \end{tikzpicture}

\end{frame}

%% Weight of the Ever Growing Data
\begin{frame} {}

    \begin{tikzpicture} [
            remember picture,
            overlay
        ]
\tikzframetitleleft{Weight of} {The Ever Growing Data}

        \begin{scope} [
                rotate = 45,
                transform canvas = {
                    yshift = -1.1cm,
                },
            ]
            \draw ($(current page.center)!0.3!(current page.east)$) coordinate (power);

            \begin{scope}
                \path [name path = mountainHandle, rounded corners = 0.7cm]
                (power) ++(-2.5, -2.5) rectangle ++(5, 5)
                ;
                \clip[use path = mountainHandle] ;
                \begin{scope}
                    \node [anchor = center] (pixel) at (power) {
                        \includegraphics[height = 0.75\paperheight]{pic/mountain_hald-clut.jpg}
                    }
                    ;
                \end{scope}
            \end{scope}

            \draw (power) ++(5.5, 0) coordinate (power);

            \filldraw [colorLightGreen, rounded corners = 0.7cm]
            (power) ++(-2.5, -2.5) rectangle ++(5, 5)
            ;

            \draw (power) ++(-5.5, -5.5) coordinate (power);

            \filldraw [rounded corners = 0.7cm]
            (power) ++(-2.5, -2.5) rectangle ++(5, 5)
            ;

            % \draw (power) ++(5.5, 0) coordinate (power);
            %
            % \filldraw [rounded corners = 0.7cm]
            % (power) ++(-2.5, -2.5) rectangle ++(5, 5)
            % ;

        \end{scope}

        \draw (frametitle.south west) ++(0, -0.3cm) coordinate (content) ;
        \node [anchor = north west]  (content) at (content) {
            \begin{minipage} {6.5cm}
                Day by day ML Models are getting more and more capabilities.
                So as the need for more and more powerful hardware to run these models.

                This can lead to several concerns:

                \begin{itemize}
                    \item Increased Carbon Footprint.
                    \item Reliance of Cloud Computing.
                    \item Latency related with Cloud Computing.
                    \item Need for powerful Hardware.
                \end{itemize}

                \alert{TinyML} is a subset of ML that is trying to address these issues.

            \end{minipage}
        }
        ;

    \end{tikzpicture}

\end{frame}

%% An Introduction to TinyML
\begin{frame} {}

    \begin{tikzpicture} [
            remember picture,
            overlay
        ]

        \tikzframetitleright{An Introduction to} {Tiny Machine Learning}

    \end{tikzpicture}
\end{frame}

\end{document}
