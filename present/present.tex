\documentclass{beamer}

% % height = 5.9in width = 4.37in
% make this variable
%% \usepackage[paperheight = 4.37in, paperwidth = 5.9in, margin=0.2in]{geometry}
\usepackage[a4paper, margin=1in]{geometry}

\usepackage[export]{adjustbox}
\usepackage{graphicx}
\usepackage{xcolor}
\usepackage{circuitikz}
\usepackage{subfiles}
\usepackage{amsmath, amssymb}
\usepackage{enumitem}
\usepackage{nicematrix}
\usepackage{minted}
\usepackage{caption}
\usepackage{lmodern}
\usepackage{bookmark}
\usepackage{tabularx}
\usepackage{multirow}
\usepackage{multicol}
\usepackage{booktabs}
\usepackage{titlesec}
\usepackage{xspace}
\usepackage{varwidth}
\usepackage{titletoc}
\usepackage{epigraph}
\usepackage{etoolbox}
\usepackage{fontawesome5}
\usepackage[style = ieee]{biblatex} %Imports biblatex package
\usepackage{xfp}
\usepackage{xifthen}
\usepackage{tcolorbox}
\usepackage{xltabular}
\usepackage[T1]{fontenc}
\usepackage{setspace}
\usepackage[numbib]{tocbibind}

%% for bold textsc to work
\rmfamily % To load Latin Modern Roman and enable the following NFSS declarations.
% Declare that Latin Modern Roman (lmr) should take
% its bold (b) and bold extended (bx) weight, and small capital (sc) shape,
% from the corresponding Computer Modern Roman (cmr) font, for the T1 font encoding.
\DeclareFontShape{T1}{lmr}{b}{sc}{<->ssub*cmr/bx/sc}{}
\DeclareFontShape{T1}{lmr}{bx}{sc}{<->ssub*cmr/bx/sc}{}

\addbibresource{abstract.bib} %Import the bibliography file

\usepackage[bottom]{footmisc}

\usepackage{gensymb}
\usepackage{siunitx}
\usepackage{pgfplots}

\pgfplotsset{
    compat=newest,
    colormap={black}{rgb255=(0,0,0) rgb255=(0,0,0)}
}

\usetikzlibrary{intersections}
\usetikzlibrary{positioning}
\usetikzlibrary{calc}
\usetikzlibrary{ext.topaths.arcthrough}
\usetikzlibrary{decorations.markings}

\setlength{\arraycolsep}{0pt}
\renewcommand\arraystretch{1.5}

%% custom packages

\usepackage{colorscheme}
\usepackage{subtikzpicture}
\usepackage{generalcommands}

% alerts

\newcommand\alertCaution[1] {
    \begin{tcolorbox}[
            coltitle = colorAlertCaution,
            colbacktitle = colorAlertBgCaution,
            colback = colorAlertBgCaution,
            colframe = colorAlertBgCaution,
            title=Caution,
            fonttitle=\bfseries,
            detach title,
        ]
        \begin{minipage}[t]{0.18\textwidth}
            \begin{flushleft}
                \tcbtitle
            \end{flushleft}
        \end{minipage}
        \begin{minipage}[t]{0.8\textwidth}
            #1
        \end{minipage}
    \end{tcolorbox}
}

\newcommand\alertWarning[1] {
    \begin{tcolorbox}[
            coltitle = colorAlertWarning,
            colbacktitle = colorAlertBgWarning,
            colback = colorAlertBgWarning,
            colframe = colorAlertBgWarning,
            title=Warning,
            fonttitle=\bfseries,
            detach title,
        ]
        \begin{minipage}[t]{0.18\textwidth}
            \begin{flushleft}
                \tcbtitle
            \end{flushleft}
        \end{minipage}
        \begin{minipage}[t]{0.8\textwidth}
            #1
        \end{minipage}
    \end{tcolorbox}
}

\newcommand\alertImportant[1] {
    \begin{tcolorbox}[
            coltitle = colorAlertImportant,
            colbacktitle = colorAlertBgImportant,
            colback = colorAlertBgImportant,
            colframe = colorAlertBgImportant,
            title=Important,
            fonttitle=\bfseries,
            detach title,
        ]
        \begin{minipage}[t]{0.18\textwidth}
            \begin{flushleft}
                \tcbtitle
            \end{flushleft}
        \end{minipage}
        \begin{minipage}[t]{0.8\textwidth}
            #1
        \end{minipage}
    \end{tcolorbox}
}

\newcommand\alertTip[1] {
    \begin{tcolorbox}[
            coltitle = colorAlertTip,
            colbacktitle = colorAlertBgTip,
            colback = colorAlertBgTip,
            colframe = colorAlertBgTip,
            title=Tip,
            fonttitle=\bfseries,
            detach title,
        ]
        \begin{minipage}[t]{0.18\textwidth}
            \begin{flushleft}
                \tcbtitle
            \end{flushleft}
        \end{minipage}
        \begin{minipage}[t]{0.8\textwidth}
            #1
        \end{minipage}
    \end{tcolorbox}
}

\newcommand\alertNote[1] {
    \begin{tcolorbox}[
            coltitle = colorAlertNote,
            colbacktitle = colorAlertBgNote,
            colback = colorAlertBgNote,
            colframe = colorAlertBgNote,
            title=Note,
            fonttitle=\bfseries,
            detach title,
        ]
        \begin{minipage}[t]{0.18\textwidth}
            \begin{flushleft}
                \tcbtitle
            \end{flushleft}
        \end{minipage}
        \begin{minipage}[t]{0.8\textwidth}
            #1
        \end{minipage}
    \end{tcolorbox}
}

%% for paragraphs
\setlength{\parskip}{0.5\baselineskip}

%% \makeatletter
%% \def\maxwidth{\ifdim\Gin@nat@width>0.8\linewidth0.8\linewidth\else\Gin@nat@width\fi}
%% \def\maxheight{\ifdim\Gin@nat@height>0.9\textheight0.9\textheight\else\Gin@nat@height\fi}
%% \makeatother
%% % Scale images if necessary, so that they will not overflow the page
%% % margins by default, and it is still possible to overwrite the defaults
%% % using explicit options in \includegraphics[width, height, ...]{}
%% \setkeys{Gin}{width=\maxwidth,height=\maxheight,keepaspectratio}
%% % Set default figure placement to htbp

%% for fontawesome

%for scalling of fontawesome
\DeclareFontFamily{U}{fontawesome1}{}
\DeclareFontShape{U}{fontawesome1}{m}{n}{<->FontAwesome--fontawesomeone}{}
\DeclareFontFamily{U}{fontawesome2}{}
\DeclareFontShape{U}{fontawesome2}{m}{n}{<->FontAwesome--fontawesometwo}{}
\DeclareFontFamily{U}{fontawesome3}{}
\DeclareFontShape{U}{fontawesome3}{m}{n}{<->FontAwesome--fontawesomethree}{}
\DeclareFontFamily{U}{fontawesome5}{}
\DeclareFontShape{U}{fontawesome5}{m}{n}{<->FontAwesome--fontawesomefive}{}
\DeclareRobustCommand{\FAone}{\usefont{U}{fontawesome1}{m}{n}}
\DeclareRobustCommand{\FAtwo}{\usefont{U}{fontawesome2}{m}{n}}
\DeclareRobustCommand{\FAthree}{\usefont{U}{fontawesome3}{m}{n}}
\DeclareRobustCommand{\FAfive}{\usefont{U}{fontawesome5}{m}{n}}

\titleformat{\chapter} [display]
{\bfseries\normalfont\huge\filright\sffamily\vspace{-2cm}}
{\Large\textsc{chapter \num[minimum-integer-digits = 2]{\thechapter}} \vspace{1em}}
{1pc}
{\titlerule\vspace{0.5em}\scshape}
[\vspace{0.5em}{\titlerule[1pt]}]

%\titleformat{\section} [display]
%{\bfseries\normalfont\large\filright\sffamily}
%{}
%{2pt}
%{\scshape}
%{}

\setlength\epigraphwidth{9cm}
\setlength\epigraphrule{0pt}

\renewcommand{\epigraphflush}{center}

%% for graphicx
%% https://tex.stackexchange.com/questions/439918/set-default-value-for-max-width-of-includegraphics

%\expandafter\patchcmd\csname Gin@ii\endcsname
%{\setkeys {Gin}{#1}}
%{%
%    \setkeys {Gin}
%    {max width = 0.8\textwidth, max height = 0.4\textwidth, keepaspectratio, #1}%
%}
%{}{}

\def\IGXMaxWidth{\textwidth}
\def\IGXMaxHeight{\textheight}
\def\IGXDefaultOptionalArgs{keepaspectratio}

\makeatletter
\def\fps@figure{htbp}
\makeatother


% \usepackage{minted}

\usepackage[size=custom,width=16,height=9,scale=0.4]{beamerposter}
% \usepackage[size=custom, width=12.8, height=9.6, scale=0.4]{beamerposter}

\usepackage[T1]{fontenc}
\usepackage{tcolorbox}
\usepackage{graphicx}
\usepackage{lmodern}
\usepackage{tabularx}
\usepackage{tikz}
\usepackage{blindtext}
\usepackage{svg}
\usepackage{fontspec}
\usepackage{booktabs}
\usepackage[style = ieee]{biblatex}
\usepackage{blindtext}
\usepackage{setspace}

\usetikzlibrary{calc}
\usetikzlibrary{intersections,decorations.markings}

\makeatletter
\tikzset{
  use path for main/.code={%
    \tikz@addmode{%
      \expandafter\pgfsyssoftpath@setcurrentpath\csname tikz@intersect@path@name@#1\endcsname
    }%
  },
  use path for actions/.code={%
    \expandafter\def\expandafter\tikz@preactions\expandafter{\tikz@preactions\expandafter\let\expandafter\tikz@actions@path\csname tikz@intersect@path@name@#1\endcsname}%
  },
  use path/.style={%
    use path for main=#1,
    use path for actions=#1,
  }
}
\setsansfont{Garamond.ttf}

% \usefonttheme{serif}
% \usepackage{beamercolorthemedracula}

% \addbibresource{abstract.bib} %Import the bibliography file

\usetheme{metropolis}

\definecolor{forestbg}      {RGB} {31, 30, 24}

\title{TinyML in Agriculture Sector}

\subtitle{Awakening of Minds in Low Power Edge Devices}

\date{\today}

\author{
    Daniel V Mathew
}

\institute{Rajiv Gandhi Institute of Technology, Kottayam}

\def\supervisor{
    Prof. Sujithamol S
}

\def\leafColorPrimaryName{Sage}
\def\leafColorPrimaryRGB{138, 184, 114}
\def\leafColorPrimaryHex{\#8AB872}
\definecolor{leafColorPrimary} {RGB} {\leafColorPrimaryRGB}
%%
\def\leafColorSecondaryName{Cactus Spike}
\def\leafColorSecondaryRGB{204, 224, 153}
\def\leafColorSecondaryHex{\#CCE099}
\definecolor{leafColorSecondary} {RGB} {\leafColorSecondaryRGB}

\definecolor{colorBlue}   {HTML} {33539E}
\definecolor{colorSky}    {HTML} {7FACD6}
\definecolor{colorViolet} {HTML} {BFB8DA}
\definecolor{colorPink}   {HTML} {E8B7D4}
\definecolor{colorMaroon} {HTML} {A5678E}

\definecolor{colorRampart} {HTML} {BDB7B7}
\definecolor{colorCaramel} {HTML} {B89B93}
\definecolor{colorCortex} {HTML} {A59691}
\definecolor{colorDove} {HTML} {B6AFA5}
\definecolor{colorCathedral} {HTML} {ABA7A6}
\definecolor{colorWind} {HTML} {CAC4C4}

\definecolor{colorRed} {HTML} {E44E54}
\definecolor{colorDarkGreen} {HTML} {2E4330}

\newcommand\tableBox[2] {
    \begin{tcolorbox}[
            coltitle = leafColorPrimary,
            colbacktitle = leafColorSecondary,
            colback = leafColorSecondary,
            colframe = leafColorSecondary,
            title=#1,
            fonttitle=\bfseries,
            detach title,
        ]
        \begin{minipage}[t]{0.1\textwidth}
            \begin{flushleft}
                \tcbtitle
            \end{flushleft}
        \end{minipage}
        \begin{minipage}[t]{0.9\textwidth}
            #2
        \end{minipage}
    \end{tcolorbox}
}

% \setbeamertemplate{background} {
%     \begin{tikzpicture} [remember picture, overlay]
%         \filldraw [
%             colorWind,
%         ]
%         (current page.north west) rectangle (current page.south east)
%         ;
%     \end{tikzpicture}
% }

% \setbeamertemplate{frametitle}{
%     \begin{tikzpicture} [remember picture, overlay]
%         \node (title) at (current page.north west) [
%             black, anchor = north west
%         ] {\strut\insertframetitle\strut};
%         \draw[
%             black,
%             very thick,
%             line cap = round,
%         ]
%         (title.east) -- (title.east -| current page.east)
%         ;
%     \end{tikzpicture}%
% }

\setbeamercolor{frametitle}{%
  fg=white,
  bg=colorMaroon
}

\newcommand\tikzframetitle[2] {
    \draw (current page.north west) ++(0.75cm, -0.75cm) coordinate (frametitle) ;
    \node [anchor = north west] (frametitle) at (frametitle) {
        \begin{minipage} {8cm}
            \setstretch{0.75}
            \huge
            {\Large #1} \\ #2
        \end{minipage}
    }
    ;
}

\colorlet{fgcol}{colorDarkGreen}

\def\markMainInnerRadius{0.175cm}
\def\markMainOuterRadius{0.3cm}
\def\markSubRadius{0.075cm}
\def\markCornerRadius{0.1cm}

\begin{document}

\maketitle

% \begin{frame} {Introduction}
%     \begin{tikzpicture}
%
%         \draw (0, 0) coordinate (tmp);
%         \foreach \x in {
%             colorDarkGreen,
%             colorRed,
%             colorBlue,
%             colorSky,
%             colorViolet,
%             colorPink,
%             colorMaroon,
%             %%
%             colorRampart,
%             colorCaramel,
%             colorCortex,
%             colorDove,
%             colorCathedral%
%         } {
%             \filldraw [ \x ] (tmp) circle (1cm) (tmp) ++(1.5, 0) coordinate (tmp);
%         }
%
%     \end{tikzpicture}
% \end{frame}

%% Making Sense of Data
\begin{frame} {}

    \begin{tikzpicture} [
            remember picture,
            overlay,
        ]

        \draw (0, 0)
        coordinate (origin)
        (current page.east) ++(1, 0) coordinate (edge)
        ;

        \begin{scope}
            %% Humidity
            \draw []
            (current page.south) ++(1.0cm, 2.5cm) coordinate (humidityMarkCenter)
            ;
            \path [name path = markHumeHandle, rounded corners = \markCornerRadius]
            (humidityMarkCenter) ++(135:\markMainOuterRadius) coordinate (humidityMarkFirst)
            (humidityMarkCenter) ++(135:(0.75cm + \markMainOuterRadius) coordinate (humidityMarkSecond)
            (humidityMarkSecond) ++(-1, 0) coordinate (humidityMarkThird)
            (humidityMarkFirst) -- (humidityMarkSecond) -- (humidityMarkThird)
            ;
            \draw [use path = markHumeHandle, very thick] ;
            ;
            \filldraw [fgcol]
            (humidityMarkThird) circle (\markSubRadius)
            ;
            \draw [fgcol]
            (humidityMarkThird) node [left = 6pt] {
                Humidity Data
            }
            ;

            %% Temperature
            \draw []
            (current page.south) ++(5.0cm, 6.55cm) coordinate (temperatureMarkCenter)
            ;
            \path [name path = markTempHandle, rounded corners = \markCornerRadius]
            (temperatureMarkCenter) ++(135:\markMainOuterRadius) coordinate (temperatureMarkFirst)
            (temperatureMarkCenter) ++(135:(0.75cm + \markMainOuterRadius) coordinate (temperatureMarkSecond)
            (temperatureMarkSecond) ++(-1, 0) coordinate (temperatureMarkThird)
            (temperatureMarkFirst) -- (temperatureMarkSecond) -- (temperatureMarkThird)
            ;
            \draw [use path = markTempHandle, very thick] ;
            ;
            \filldraw [fgcol]
            (temperatureMarkThird) circle (\markSubRadius)
            ;
            \draw [fgcol]
            (temperatureMarkThird) node [left = 6pt] {
                Temperature Data
            }
            ;

            \clip [rotate = 45, rounded corners = 0.7cm]
            (edge) ++(-10, -15) rectangle ++(20, 20)
            ;

            \begin{scope}
                \node [anchor = east] (img) at (edge) {
                    \includegraphics[height = \paperheight]{low.jpg}
                }
                ;

                %% Humidity
                \filldraw [white!90!fgcol]
                (humidityMarkCenter) circle (\markMainInnerRadius)
                ;
                \draw [white!90!fgcol, very thick]
                (humidityMarkCenter) circle (\markMainOuterRadius)
                ;
                \draw [use path = markHumeHandle, white!90!fgcol, very thick] ;

                %% Temperature
                \filldraw [white!90!fgcol]
                (temperatureMarkCenter) circle (\markMainInnerRadius)
                ;
                \draw [white!90!fgcol, very thick]
                (temperatureMarkCenter) circle (\markMainOuterRadius)
                ;
                \draw [use path = markTempHandle, white!90!fgcol, very thick] ;

            \end{scope}
        \end{scope}

        \tikzframetitle{Making} {Sense of Data}

        \draw (frametitle.south west) ++(0, -0.5cm) coordinate (content) ;
        \node [anchor = north west]  (content) at (content) {
            \begin{minipage} {6cm}
                \begin{itemize}[]
                    \item Say something.
                    \item Interesting.
                \end{itemize}
            \end{minipage}
        }
        ;

        \draw (0, 0)
        coordinate (origin)
        (current page.west) coordinate (edge)
        ;
        \begin{scope} [transform canvas = {yshift = -2.2cm, xshift = 0.5cm}]
            \draw []
            (edge) ++(2cm, -0.5cm) coordinate (pixelMarkCenter)
            ;
            \path [name path = markHandle, rounded corners = \markCornerRadius]
            (pixelMarkCenter) ++(-45:\markMainOuterRadius) coordinate (pixelMarkFirst)
            (pixelMarkCenter) ++(-45:(1cm + \markMainOuterRadius) coordinate (pixelMarkSecond)
            (pixelMarkSecond) ++(1, 0) coordinate (pixelMarkThird)
            (pixelMarkFirst) -- (pixelMarkSecond) -- (pixelMarkThird)
            ;
            \draw [use path = markHandle, very thick] ;
            \filldraw [fgcol]
            (pixelMarkThird) circle (\markSubRadius)
            ;
            \draw [fgcol]
            (pixelMarkThird) node [right = 6pt] {
                Pixel Data
            }
            ;
            \clip [rotate = -45, rounded corners = 0.7cm]
            (edge) rectangle ++(2.5, 2.5)
            ;
            \begin{scope}
                \node [anchor = west] (pixel) at (edge) {
                    \includegraphics[height = \paperheight]{low.jpg}
                }
                ;
                \draw [thin, step = 0.1, white, opacity = 0.40]
                (edge) ++(-1, -2) grid ++(5, 5)
                ;
                \filldraw [white!90!fgcol]
                (pixelMarkCenter) circle (\markMainInnerRadius)
                ;

                \draw [white!90!fgcol, very thick]
                (pixelMarkCenter) circle (\markMainOuterRadius)
                ;
                \draw [use path = markHandle, white!90!fgcol, very thick] ;
            \end{scope}
        \end{scope}

    \end{tikzpicture}

\end{frame}

% \begin{frame} {Learning to Ask the Right Questions}
% \end{frame}
%
% \begin{frame} {Matrices are just Consdened If-Elses}
% \end{frame}
%
% \begin{frame} {Guess Who's Good at Crunching Matrices?}
% \end{frame}
%
% \begin{frame} {Weight of the Ever Growing Data}
% \end{frame}
%
% \begin{frame} {TinyML: Introduction}
% \end{frame}

\end{document}
