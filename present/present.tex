\documentclass{beamer}

\usepackage[size=custom,width=16,height=9,scale=0.4]{beamerposter}
% \usepackage[size=custom, width=12.8, height=9.6, scale=0.4]{beamerposter}

% height = 5.9in width = 4.37in
% make this variable
%% \usepackage[paperheight = 4.37in, paperwidth = 5.9in, margin=0.2in]{geometry}
\usepackage[a4paper, margin=1in]{geometry}

\usepackage[export]{adjustbox}
\usepackage{graphicx}
\usepackage{xcolor}
\usepackage{circuitikz}
\usepackage{subfiles}
\usepackage{amsmath, amssymb}
\usepackage{enumitem}
\usepackage{nicematrix}
\usepackage{minted}
\usepackage{caption}
\usepackage{lmodern}
\usepackage{bookmark}
\usepackage{tabularx}
\usepackage{multirow}
\usepackage{multicol}
\usepackage{booktabs}
\usepackage{titlesec}
\usepackage{xspace}
\usepackage{varwidth}
\usepackage{titletoc}
\usepackage{epigraph}
\usepackage{etoolbox}
\usepackage{fontawesome5}
\usepackage[style = ieee]{biblatex} %Imports biblatex package
\usepackage{xfp}
\usepackage{xifthen}
\usepackage{tcolorbox}
\usepackage{xltabular}
\usepackage[T1]{fontenc}
\usepackage{setspace}
\usepackage[numbib]{tocbibind}

%% for bold textsc to work
\rmfamily % To load Latin Modern Roman and enable the following NFSS declarations.
% Declare that Latin Modern Roman (lmr) should take
% its bold (b) and bold extended (bx) weight, and small capital (sc) shape,
% from the corresponding Computer Modern Roman (cmr) font, for the T1 font encoding.
\DeclareFontShape{T1}{lmr}{b}{sc}{<->ssub*cmr/bx/sc}{}
\DeclareFontShape{T1}{lmr}{bx}{sc}{<->ssub*cmr/bx/sc}{}

\addbibresource{abstract.bib} %Import the bibliography file

\usepackage[bottom]{footmisc}

\usepackage{gensymb}
\usepackage{siunitx}
\usepackage{pgfplots}

\pgfplotsset{
    compat=newest,
    colormap={black}{rgb255=(0,0,0) rgb255=(0,0,0)}
}

\usetikzlibrary{intersections}
\usetikzlibrary{positioning}
\usetikzlibrary{calc}
\usetikzlibrary{ext.topaths.arcthrough}
\usetikzlibrary{decorations.markings}

\setlength{\arraycolsep}{0pt}
\renewcommand\arraystretch{1.5}

%% custom packages

\usepackage{colorscheme}
\usepackage{subtikzpicture}
\usepackage{generalcommands}

% alerts

\newcommand\alertCaution[1] {
    \begin{tcolorbox}[
            coltitle = colorAlertCaution,
            colbacktitle = colorAlertBgCaution,
            colback = colorAlertBgCaution,
            colframe = colorAlertBgCaution,
            title=Caution,
            fonttitle=\bfseries,
            detach title,
        ]
        \begin{minipage}[t]{0.18\textwidth}
            \begin{flushleft}
                \tcbtitle
            \end{flushleft}
        \end{minipage}
        \begin{minipage}[t]{0.8\textwidth}
            #1
        \end{minipage}
    \end{tcolorbox}
}

\newcommand\alertWarning[1] {
    \begin{tcolorbox}[
            coltitle = colorAlertWarning,
            colbacktitle = colorAlertBgWarning,
            colback = colorAlertBgWarning,
            colframe = colorAlertBgWarning,
            title=Warning,
            fonttitle=\bfseries,
            detach title,
        ]
        \begin{minipage}[t]{0.18\textwidth}
            \begin{flushleft}
                \tcbtitle
            \end{flushleft}
        \end{minipage}
        \begin{minipage}[t]{0.8\textwidth}
            #1
        \end{minipage}
    \end{tcolorbox}
}

\newcommand\alertImportant[1] {
    \begin{tcolorbox}[
            coltitle = colorAlertImportant,
            colbacktitle = colorAlertBgImportant,
            colback = colorAlertBgImportant,
            colframe = colorAlertBgImportant,
            title=Important,
            fonttitle=\bfseries,
            detach title,
        ]
        \begin{minipage}[t]{0.18\textwidth}
            \begin{flushleft}
                \tcbtitle
            \end{flushleft}
        \end{minipage}
        \begin{minipage}[t]{0.8\textwidth}
            #1
        \end{minipage}
    \end{tcolorbox}
}

\newcommand\alertTip[1] {
    \begin{tcolorbox}[
            coltitle = colorAlertTip,
            colbacktitle = colorAlertBgTip,
            colback = colorAlertBgTip,
            colframe = colorAlertBgTip,
            title=Tip,
            fonttitle=\bfseries,
            detach title,
        ]
        \begin{minipage}[t]{0.18\textwidth}
            \begin{flushleft}
                \tcbtitle
            \end{flushleft}
        \end{minipage}
        \begin{minipage}[t]{0.8\textwidth}
            #1
        \end{minipage}
    \end{tcolorbox}
}

\newcommand\alertNote[1] {
    \begin{tcolorbox}[
            coltitle = colorAlertNote,
            colbacktitle = colorAlertBgNote,
            colback = colorAlertBgNote,
            colframe = colorAlertBgNote,
            title=Note,
            fonttitle=\bfseries,
            detach title,
        ]
        \begin{minipage}[t]{0.18\textwidth}
            \begin{flushleft}
                \tcbtitle
            \end{flushleft}
        \end{minipage}
        \begin{minipage}[t]{0.8\textwidth}
            #1
        \end{minipage}
    \end{tcolorbox}
}

%% for paragraphs
\setlength{\parskip}{0.5\baselineskip}

%% \makeatletter
%% \def\maxwidth{\ifdim\Gin@nat@width>0.8\linewidth0.8\linewidth\else\Gin@nat@width\fi}
%% \def\maxheight{\ifdim\Gin@nat@height>0.9\textheight0.9\textheight\else\Gin@nat@height\fi}
%% \makeatother
%% % Scale images if necessary, so that they will not overflow the page
%% % margins by default, and it is still possible to overwrite the defaults
%% % using explicit options in \includegraphics[width, height, ...]{}
%% \setkeys{Gin}{width=\maxwidth,height=\maxheight,keepaspectratio}
%% % Set default figure placement to htbp

%% for fontawesome

%for scalling of fontawesome
\DeclareFontFamily{U}{fontawesome1}{}
\DeclareFontShape{U}{fontawesome1}{m}{n}{<->FontAwesome--fontawesomeone}{}
\DeclareFontFamily{U}{fontawesome2}{}
\DeclareFontShape{U}{fontawesome2}{m}{n}{<->FontAwesome--fontawesometwo}{}
\DeclareFontFamily{U}{fontawesome3}{}
\DeclareFontShape{U}{fontawesome3}{m}{n}{<->FontAwesome--fontawesomethree}{}
\DeclareFontFamily{U}{fontawesome5}{}
\DeclareFontShape{U}{fontawesome5}{m}{n}{<->FontAwesome--fontawesomefive}{}
\DeclareRobustCommand{\FAone}{\usefont{U}{fontawesome1}{m}{n}}
\DeclareRobustCommand{\FAtwo}{\usefont{U}{fontawesome2}{m}{n}}
\DeclareRobustCommand{\FAthree}{\usefont{U}{fontawesome3}{m}{n}}
\DeclareRobustCommand{\FAfive}{\usefont{U}{fontawesome5}{m}{n}}

\titleformat{\chapter} [display]
{\bfseries\normalfont\huge\filright\sffamily\vspace{-2cm}}
{\Large\textsc{chapter \num[minimum-integer-digits = 2]{\thechapter}} \vspace{1em}}
{1pc}
{\titlerule\vspace{0.5em}\scshape}
[\vspace{0.5em}{\titlerule[1pt]}]

%\titleformat{\section} [display]
%{\bfseries\normalfont\large\filright\sffamily}
%{}
%{2pt}
%{\scshape}
%{}

\setlength\epigraphwidth{9cm}
\setlength\epigraphrule{0pt}

\renewcommand{\epigraphflush}{center}

%% for graphicx
%% https://tex.stackexchange.com/questions/439918/set-default-value-for-max-width-of-includegraphics

%\expandafter\patchcmd\csname Gin@ii\endcsname
%{\setkeys {Gin}{#1}}
%{%
%    \setkeys {Gin}
%    {max width = 0.8\textwidth, max height = 0.4\textwidth, keepaspectratio, #1}%
%}
%{}{}

\def\IGXMaxWidth{\textwidth}
\def\IGXMaxHeight{\textheight}
\def\IGXDefaultOptionalArgs{keepaspectratio}

\makeatletter
\def\fps@figure{htbp}
\makeatother


\usepackage{fontspec}

\title{TinyML in Agriculture Sector}

\subtitle{Awakening of Minds in Low Power Edge Devices}

\author{
    Daniel V Mathew \\ KTE22EC029
}

\institute{Rajiv Gandhi Institute of Technology, Kottayam}

\def\supervisor{
    Prof. Sujithamol S
}

\date{Guided by: \supervisor}

\renewcommand\alert[1] {%
    {\color{colorRed}#1}%
}

\usetheme{metropolis}

% \renewcommand\arraystretch{1.75}
% \renewcommand\midrule{{\color{colorRed} \vspace{-8pt}\rule{\textwidth}{0.75pt}\vspace{8pt}}}

% \setbeamertemplate{background} {
%     \begin{tikzpicture} [remember picture, overlay]
%         \filldraw [
%             colorWind,
%         ]
%         (current page.north west) rectangle (current page.south east)
%         ;
%     \end{tikzpicture}
% }

% \setbeamertemplate{frametitle}{
%     \begin{tikzpicture} [remember picture, overlay]
%         \node (title) at (current page.north west) [
%             black, anchor = north west
%         ] {\strut\insertframetitle\strut};
%         \draw[
%             black,
%             very thick,
%             line cap = round,
%         ]
%         (title.east) -- (title.east -| current page.east)
%         ;
%     \end{tikzpicture}%
% }

\setbeamercolor{frametitle}{%
  fg=white,
  bg=colorMaroon
}

% \setsansfont[ItalicFont={Georgia.ttf}, BoldFont={Georgia.ttf}]{Garamond.ttf}%

\setsansfont[
    ItalicFont={Garamond Italic.ttf},%
    BoldFont={Garamond Bold.ttf},%
    BoldItalicFont={Garamond Bold Italic.ttf}%
] {Garamond.ttf}%

% \setsansfont{Garamond Bold.ttf}
% \setsansfont{Georgia.ttf}

% \usefonttheme{serif}
% \usepackage{beamercolorthemedracula}

% \addbibresource{abstract.bib} %Import the bibliography file

\def\squareSide{3.5cm}
\def\squareGap{0.3cm}

\subtikzpicturedef{subSquare} {
    rightCorner,
    leftCorner,
    botCorner,
    topCorner,
    padRightCorner,
    padLeftCorner,
    padBotCorner,
    padTopCorner,
    center,
    origin%
} {
    \draw (#1-start) coordinate (#1-origin);
    \draw
    (#1-origin) coordinate (#1-botCorner)
    %%
    (#1-botCorner) ++(45:\squareSide) coordinate (#1-rightCorner)
    (#1-botCorner) ++(135:\squareSide) coordinate (#1-leftCorner)
    ($(#1-leftCorner)!0.5!(#1-rightCorner)$) coordinate (#1-center)
    %%
    (#1-rightCorner) ++(135:\squareSide) coordinate (#1-topCorner)
    %%
    (#1-rightCorner) ++(\squareGap, 0)  coordinate (#1-padRightCorner)
    (#1-leftCorner) ++(-\squareGap, 0)   coordinate (#1-padLeftCorner)
    (#1-botCorner) ++(0, -\squareGap)    coordinate (#1-padBotCorner)
    (#1-topCorner) ++(0, \squareGap)    coordinate (#1-padTopCorner)
    ;
}

\subtikzpictureactivate{subSquare}

\newcommand\drawSubSquare[2] {
    \filldraw [
        #1%
        rounded corners = 0.7cm,
    ]
    (#2-botCorner) -- (#2-rightCorner) -- (#2-topCorner) -- (#2-leftCorner) -- cycle
    ;
}

\begin{document}

\maketitle

% \begin{frame} {Introduction}
%     \begin{tikzpicture}
%
%         \draw (0, 0) coordinate (tmp);
%         \foreach \x in {
%             colorDarkGreen,
%             colorRed,
%             colorBlue,
%             colorSky,
%             colorViolet,
%             colorPink,
%             colorMaroon,
%             %%
%             colorRampart,
%             colorCaramel,
%             colorCortex,
%             colorDove,
%             colorCathedral%
%         } {
%             \filldraw [ \x ] (tmp) circle (1cm) (tmp) ++(1.5, 0) coordinate (tmp);
%         }
%
%     \end{tikzpicture}
% \end{frame}

%% Making Sense of Data
\begin{frame} {}
    \begin{tikzpicture} [
            remember picture,
            overlay,
        ]

        \draw (0, 0)
        coordinate (origin)
        (current page.east) ++(1, 0) coordinate (edge)
        ;

        \begin{scope}
            %% Humidity
            \draw []
            (current page.south) ++(1.0cm, 2.5cm) coordinate (humidityMarkCenter)
            ;
            \path [name path = markHumeHandle, rounded corners = \markCornerRadius]
            (humidityMarkCenter) ++(135:\markMainOuterRadius) coordinate (humidityMarkFirst)
            (humidityMarkCenter) ++(135:(0.75cm + \markMainOuterRadius) coordinate (humidityMarkSecond)
            (humidityMarkSecond) ++(-1, 0) coordinate (humidityMarkThird)
            (humidityMarkFirst) -- (humidityMarkSecond) -- (humidityMarkThird)
            ;
            \draw [use path = markHumeHandle, very thick] ;
            ;
            \filldraw [fgcol]
            (humidityMarkThird) circle (\markSubRadius)
            ;
            \draw [fgcol]
            (humidityMarkThird) node [left = 6pt] {
                Humidity Data
            }
            ;

            %% Temperature
            \draw []
            (current page.south) ++(5.0cm, 6.55cm) coordinate (temperatureMarkCenter)
            ;
            \path [name path = markTempHandle, rounded corners = \markCornerRadius]
            (temperatureMarkCenter) ++(135:\markMainOuterRadius) coordinate (temperatureMarkFirst)
            (temperatureMarkCenter) ++(135:(0.75cm + \markMainOuterRadius) coordinate (temperatureMarkSecond)
            (temperatureMarkSecond) ++(-1, 0) coordinate (temperatureMarkThird)
            (temperatureMarkFirst) -- (temperatureMarkSecond) -- (temperatureMarkThird)
            ;
            \draw [use path = markTempHandle, very thick] ;
            ;
            \filldraw [fgcol]
            (temperatureMarkThird) circle (\markSubRadius)
            ;
            \draw [fgcol]
            (temperatureMarkThird) node [left = 6pt] {
                Temperature Data
            }
            ;

            \clip [rotate = 45, rounded corners = 0.7cm]
            (edge) ++(-10, -15) rectangle ++(20, 20)
            ;

            \begin{scope}
                \node [anchor = east] (img) at (edge) {
                    \includegraphics[height = \paperheight]{pics/red_forest.png}
                }
                ;

                %% Humidity
                \filldraw [white!90!fgcol]
                (humidityMarkCenter) circle (\markMainInnerRadius)
                ;
                \draw [white!90!fgcol, very thick]
                (humidityMarkCenter) circle (\markMainOuterRadius)
                ;
                \draw [use path = markHumeHandle, white!90!fgcol, very thick] ;

                %% Temperature
                \filldraw [white!90!fgcol]
                (temperatureMarkCenter) circle (\markMainInnerRadius)
                ;
                \draw [white!90!fgcol, very thick]
                (temperatureMarkCenter) circle (\markMainOuterRadius)
                ;
                \draw [use path = markTempHandle, white!90!fgcol, very thick] ;

            \end{scope}
        \end{scope}

        \tikzframetitleleft{Making} {Sense of Data}

        \draw (frametitle.south west) ++(0, -0.5cm) coordinate (content) ;
        \node [anchor = north west]  (content) at (content) {
            \begin{minipage} {8.75cm}

                Every system is in some sense a \alert{Data Processing System}.
                They \alert{Convert / Extract / Manipulate} Data in different ways.
                There are several kinds of data, such as \alert{Visual}, \alert{Auditory},
                \alert{Sensory}, etc. Goal of almost any system is to make the most use of
                the data available to it.

            \end{minipage}
        }
        ;

        \draw (0, 0)
        coordinate (origin)
        (current page.west) coordinate (edge)
        ;
        \begin{scope} [transform canvas = {yshift = -2.2cm, xshift = 0.5cm}]
            \draw []
            (edge) ++(2cm, -0.5cm) coordinate (pixelMarkCenter)
            ;
            \path [name path = markHandle, rounded corners = \markCornerRadius]
            (pixelMarkCenter) ++(-45:\markMainOuterRadius) coordinate (pixelMarkFirst)
            (pixelMarkCenter) ++(-45:(1cm + \markMainOuterRadius) coordinate (pixelMarkSecond)
            (pixelMarkSecond) ++(1, 0) coordinate (pixelMarkThird)
            (pixelMarkFirst) -- (pixelMarkSecond) -- (pixelMarkThird)
            ;
            \draw [use path = markHandle, very thick] ;
            \filldraw [fgcol]
            (pixelMarkThird) circle (\markSubRadius)
            ;
            \draw [fgcol]
            (pixelMarkThird) node [right = 6pt] {
                Pixel Data
            }
            ;
            \clip [rotate = -45, rounded corners = 0.7cm]
            (edge) rectangle ++(2.5, 2.5)
            ;
            \begin{scope}
                \node [anchor = west] (pixel) at (edge) {
                    \includegraphics[height = \paperheight]{pics/red_forest.png}
                }
                ;
                \draw [thin, step = 0.1, white, opacity = 0.40]
                (edge) ++(-1, -2) grid ++(5, 5)
                ;
                \filldraw [white!90!fgcol]
                (pixelMarkCenter) circle (\markMainInnerRadius)
                ;

                \draw [white!90!fgcol, very thick]
                (pixelMarkCenter) circle (\markMainOuterRadius)
                ;
                \draw [use path = markHandle, white!90!fgcol, very thick] ;
            \end{scope}
        \end{scope}

    \end{tikzpicture}

\end{frame}

%% Learning to Ask the Right Questions
\begin{frame} {}

    \begin{tikzpicture} [
            remember picture,
            overlay,
        ]
        \tikzframetitleleft{Asking}{The Right Questions}

        \draw (frametitle.south west) ++(0, -0.5cm) coordinate (content) ;
        \node [anchor = north west]  (content) at (content) {
            \begin{minipage} {0.45\paperwidth}
                In order to make sense of the data, we need to ask the right questions about the data.
                The sequence of these questions is known as an \alert{Algorithm}.
            \end{minipage}
        }
        ;

        \filldraw [colorTest]
        (current page.north east) ++(-7,0) coordinate (workFlowTop)
        (workFlowTop) rectangle (current page.south east)
        ;

        \draw (workFlowTop) ++(0, -0.2) coordinate (tmp);
        \foreach \x in {1, 2, 3, 4} {
            \draw [very thick, white, rounded corners = 0.2cm, line cap = round]
            (tmp) ++(0, -1) -- ++(3.5, -0.8) -- ++(3.5, 0.8)
            (tmp) ++(0, -0.25\paperheight) coordinate (tmp)
            ;
        }

        \draw ($(workFlowTop)!0.5!(current page.north east)$) ++(0, -0.8) coordinate (tmp);
        \foreach \x in {
            Finding a Problem,
            Selecting an Algorithm,
            Implementing the Algorithm,
            Solving the Problem%
        } {
            \draw [ultra thick, white, rounded corners = 0.2cm]
            (tmp)
            node [anchor = center] {
                \begin{minipage} {7cm}
                    \centering
                    \x
                \end{minipage}
            }
            % (tmp) ++(0, -1) -- ++(3.5, -1.25) -- ++(3.5, 1.25)
            (tmp) ++(0, -0.25\paperheight) coordinate (tmp)
            ;
        }

        %% Methodology

        \begin{scope}
            \draw
            (current page.center) ++(1.75cm, 0) coordinate (workFlowMarkCenter)
            % (current page.center) ++(1.6cm, 0.2cm) coordinate (workFlowMarkCenter)
            ;
            \path [name path = markWorkHandle, rounded corners = \markCornerRadius]
            (workFlowMarkCenter) ++(-180:\markMainOuterRadius) coordinate (workFlowMarkFirst)
            (workFlowMarkFirst) ++(-1, 0) coordinate (workFlowMarkSecond)
            (workFlowMarkSecond) ++(-135:0.75cm) coordinate (workFlowMarkThird)
            (workFlowMarkFirst) -- (workFlowMarkSecond) -- (workFlowMarkThird)
            ;
            \draw [use path = markWorkHandle, very thick] ;
            \filldraw [fgcol]
            (workFlowMarkThird) circle (\markSubRadius)
            ;
            \draw [fgcol]
            (workFlowMarkThird) node (workFlowMark) [left = 6pt] {
                Typical Way of Solving a Problem
            }
            ;
            \clip (workFlowTop) rectangle (current page.south east);
            \begin{scope}
                \filldraw [white!90!fgcol]
                (workFlowMarkCenter) circle (\markMainInnerRadius)
                ;
                \draw [white!90!fgcol, very thick]
                (workFlowMarkCenter) circle (\markMainOuterRadius)
                ;
                \draw [white!90!fgcol, use path = markWorkHandle, very thick] ;
            \end{scope}
        \end{scope}

        \draw (content.south west |- workFlowMark.south west) ++(0, -0.5cm) coordinate (content) ;
        \node [anchor = north west]  (content) at (content) {
            \begin{minipage} {0.45\paperwidth}
                But what if the sheer number of questions one must ask to get to the right answer gets
                so overwhelming?
            \end{minipage}
        }
        ;

        \draw (content.south west) ++(0, -0.5cm) coordinate (content) ;
        \node [anchor = north west]  (content) at (content) {
            \begin{minipage} {0.45\paperwidth}
                But before that...
            \end{minipage}
        }
        ;

    \end{tikzpicture}

\end{frame}

%% Ways of Asking Questions
\begin{frame}[fragile] {}

    \begin{tikzpicture} [
            remember picture,
            overlay
        ]
        \tikzframetitleright {Ways of} {Asking Questions}

        \begin{scope} [transform canvas = {xshift = 0.2cm}]
            \begin{scope}
                \draw
                (current page.north) ++(-0.75, -0.75) coordinate (ifelseMarkCenter)
                ;
                \path [name path = ifelseHandle, rounded corners = \markCornerRadius]
                (ifelseMarkCenter) ++(-45:\markMainOuterRadius) coordinate (ifelseMarkFirst)
                (ifelseMarkCenter) ++(-45:(2.85cm + \markMainOuterRadius) coordinate (ifelseMarkSecond)
                (ifelseMarkSecond) ++(1.4, 0) coordinate (ifelseMarkThird)
                (ifelseMarkFirst) -- (ifelseMarkSecond) -- (ifelseMarkThird)
                ;
                \draw [use path = ifelseHandle, very thick] ;
                \filldraw [
                    colorLightGreen,
                    rounded corners = 0.7cm,
                ]
                (current page.north west) ++(0, 0.1cm) ++(0,2) coordinate (tmpStart)
                (current page.center) ++(0, 0.1cm) coordinate (tmpEnd)
                (tmpStart) rectangle (tmpEnd)
                ;
                \filldraw [fgcol]
                (ifelseMarkThird) circle (\markSubRadius)
                ;
                \draw [fgcol]
                (ifelseMarkThird) node (ifelse) [right = 6pt] {
                    Using If-Else Statements
                }
                ;
                \clip [
                    rounded corners = 0.7cm,
                ]
                (current page.north west) ++(0, 0.1cm) ++(0,2) coordinate (tmpStart)
                (current page.center) ++(0, 0.1cm) coordinate (tmpEnd)
                (tmpStart) rectangle (tmpEnd)
                ;
                \begin{scope}
                    \filldraw [white!90!fgcol]
                    (ifelseMarkCenter) circle (\markMainInnerRadius)
                    ;
                    \draw [white!90!fgcol, very thick]
                    (ifelseMarkCenter) circle (\markMainOuterRadius)
                    ;
                    \draw [white!90!fgcol, use path = ifelseHandle, very thick] ;
                \end{scope}
            \end{scope}

            % \draw [thin, step = 0.1, white, opacity = 0.40]
            % (current page.north west) ++(-2,2) grid (current page.center)
            % ;

            \begin{scope}
                \draw
                (current page.south) ++(-0.75, 1.0) coordinate (matrixMarkCenter)
                ;
                \path [name path = matrixHandle, rounded corners = \markCornerRadius]
                (matrixMarkCenter) ++(45:\markMainOuterRadius) coordinate (matrixMarkFirst)
                (matrixMarkCenter) ++(45:(2.0cm + \markMainOuterRadius) coordinate (matrixMarkSecond)
                (matrixMarkSecond) ++(1.35, 0) coordinate (matrixMarkThird)
                (matrixMarkFirst) -- (matrixMarkSecond) -- (matrixMarkThird)
                ;
                \draw [use path = matrixHandle, very thick] ;
                \filldraw [
                    colorLightSageGreen,
                    rounded corners = 0.7cm,
                ]
                (current page.south west) ++(0, -0.1cm) ++(0,-2) coordinate (tmpStart)
                (current page.center) ++(0, -0.1cm) coordinate (tmpEnd)
                (tmpStart) rectangle (tmpEnd)
                ;
                \filldraw [fgcol]
                (matrixMarkThird) circle (\markSubRadius)
                ;
                \draw [fgcol]
                (matrixMarkThird) node (matrix) [right = 6pt] {
                    Using Matrix Transformation
                }
                ;
                \clip [
                    rounded corners = 0.7cm,
                ]
                (current page.south west) ++(0, -0.1cm) ++(0,-2) coordinate (tmpStart)
                (current page.center) ++(0, -0.1cm) coordinate (tmpEnd)
                (tmpStart) rectangle (tmpEnd)
                ;
                \begin{scope}
                    \filldraw [white!90!fgcol]
                    (matrixMarkCenter) circle (\markMainInnerRadius)
                    ;
                    \draw [white!90!fgcol, very thick]
                    (matrixMarkCenter) circle (\markMainOuterRadius)
                    ;
                    \draw [white!90!fgcol, use path = matrixHandle, very thick] ;
                \end{scope}
            \end{scope}

            % \draw [thin, step = 0.1, white, opacity = 0.40]
            % (current page.south west) ++(-2,-2) grid (current page.center)
            % ;

            \draw (current page.north west) ++(0.5cm, -0.5cm) coordinate (content) ;
            \node [anchor = north west]  (content) at (content) {
                \begin{minipage} {0.55\paperwidth}
                    \begin{minted}[breaklines, autogobble, mathescape] {text}
                        if this_pixel == this_color:
                            if this_pixel is at_edge:
                                # more and more if statements
                        elif this_pixel == that_color:
                            if this_pixel is in_the_middle:
                                # many more
                    \end{minted}
                \end{minipage}
            }
            ;

            \draw (current page.south west) ++(0.5cm, 0.5cm) coordinate (content) ;
            \node [anchor = south west]  (content) at (content) {
                \begin{minipage} {0.425\paperwidth}
                    \centering
                    $\begin{pNiceMatrix}
                        c_{11} & c_{12} & c_{13} & c_{14} & \cdots & c_{1n} \\
                        c_{21} & c_{22} & c_{23} & c_{24} & \cdots & c_{2n} \\
                        c_{31} & c_{32} & c_{33} & c_{34} & \cdots & c_{3n} \\
                        c_{41} & c_{42} & c_{43} & c_{44} & \cdots & c_{4n} \\
                        \vdots & \vdots & \vdots & \vdots & \ddots & \vdots\\
                        c_{m1} & c_{m2} & c_{m3} & c_{m4} & \cdots & c_{mn} \\
                    \end{pNiceMatrix}$
                \end{minipage}
            }
            ;

        \end{scope}

        \draw (ifelse.south east) ++(0.2cm, -0.5cm) coordinate (content) ;
        \node [anchor = north east]  (content) at (content) {
            \begin{minipage} {6.5cm}
                There are several ways to ask the same question. One might be able to
                to write an Algorithm for parsing the pixel data just by using \alert{If-Else}
                statements.
            \end{minipage}
        }
        ;

        \draw (matrix.south east) ++(0.2cm, -0.5cm) coordinate (content) ;
        \node [anchor = north east]  (content) at (content) {
            \begin{minipage} {6.5cm}
                Or one can directly transform input into output using a \alert{Matrix Tranformation}.
            \end{minipage}
        }
        ;

    \end{tikzpicture}

\end{frame}

%% Guess Who's Good at Crunching Matrices?
\begin{frame} {}

    \begin{tikzpicture} [
            remember picture,
            overlay
        ]

        \tikzframetitleright{Guess Who's} {Good at Crunching Matrices?}

        \draw (0, 0)
        coordinate (origin)
        (current page.south west) coordinate (edge)
        ;

        \begin{scope} [
                transform canvas = {
                    yshift = -13,
                    xshift = -13,
                }
            ]
            \begin{scope} [
                    transform canvas = {
                        yshift = 0.1cm,
                        xshift = 0.1cm,
                    }
                ]
                \draw
                (current page.north) ++(-2.0, -1.6) coordinate (cpuMarkCenter)
                ;
                \path [name path = cpuHandle, rounded corners = \markCornerRadius]
                (cpuMarkCenter) ++(0:\markMainOuterRadius) coordinate (cpuMarkFirst)
                (cpuMarkCenter) ++(0:(1.5cm + \markMainOuterRadius) coordinate (cpuMarkSecond)
                (cpuMarkSecond) ++(-45:2.1cm) coordinate (cpuMarkThird)
                (cpuMarkThird) ++(5, 0) coordinate (cpuMarkFourth)
                (cpuMarkFirst) -- (cpuMarkSecond) -- (cpuMarkThird) -- (cpuMarkFourth)
                ;
                \draw [use path = cpuHandle, very thick] ;
                \filldraw [fgcol]
                (cpuMarkFourth) circle (\markSubRadius)
                ;
                \draw [fgcol]
                (cpuMarkFourth) node (cpu) [right = 6pt] {
                    CPUs
                }
                ;
                \begin{scope}
                    \clip [rotate = 45, rounded corners = 0.7cm]
                    (edge) ++(0.25, 0.25) rectangle ++(10, 10)
                    ;
                    \draw
                    (edge) ++(3.00, 0.35) coordinate (tmp)
                    ;
                    \node [anchor = south] (img) at (tmp) {
                        \includegraphics[height = \paperheight]{pics/cpu.png}
                    }
                    ;
                    \filldraw [white!90!fgcol]
                    (cpuMarkCenter) circle (\markMainInnerRadius)
                    ;
                    \draw [white!90!fgcol, very thick]
                    (cpuMarkCenter) circle (\markMainOuterRadius)
                    ;
                    \draw [white!90!fgcol, use path = cpuHandle, very thick] ;
                \end{scope}

                \draw (cpu.south east) ++(0, -0.3cm) coordinate (content) ;
                \node [anchor = north east]  (content) at (content) {
                    \begin{minipage} {6.0cm}
                        \raggedleft
                        \alert{CPUs} are much more capable at processing matrices that anyone of us, humans.
                        % \alert{Computers}, especially \alert{GPUs} can crunch through matrices.
                    \end{minipage}
                }
                ;
            \end{scope}

            \begin{scope} [
                    transform canvas = {
                        yshift = -0.5cm,
                        xshift = -0.5cm,
                    }
                ]
                \draw
                (current page.south) ++(1.75, 3.25) coordinate (gpuMarkCenter)
                ;
                \path [name path = gpuHandle, rounded corners = \markCornerRadius]
                (gpuMarkCenter) ++(45:\markMainOuterRadius) coordinate (gpuMarkFirst)
                (gpuMarkCenter) ++(45:(1.0cm + \markMainOuterRadius) coordinate (gpuMarkSecond)
                (gpuMarkSecond) ++(4.25, 0) coordinate (gpuMarkThird)
                (gpuMarkFirst) -- (gpuMarkSecond) -- (gpuMarkThird)
                ;
                \draw [use path = gpuHandle, very thick] ;
                \filldraw [fgcol]
                (gpuMarkThird) circle (\markSubRadius)
                ;
                \draw [fgcol]
                (gpuMarkThird) node (gpu) [right = 6pt] {
                    GPUs
                }
                ;
                \begin{scope}
                    \clip [rotate = 45, rounded corners = 0.7cm]
                    (edge) ++(0.25, -0.25) rectangle ++(10, -10)
                    ;
                    \draw
                    (edge) ++(0, -0.5) coordinate (tmp)
                    ;
                    \node [anchor = south west] (img) at (tmp) {
                        \includegraphics[height = 0.9\paperheight]{pics/gpu.png}
                    }
                    ;
                    \filldraw [white!90!fgcol]
                    (gpuMarkCenter) circle (\markMainInnerRadius)
                    ;
                    \draw [white!90!fgcol, very thick]
                    (gpuMarkCenter) circle (\markMainOuterRadius)
                    ;
                    \draw [white!90!fgcol, use path = gpuHandle, very thick] ;
                \end{scope}
                \draw (gpu.south east) ++(0, -0.3cm) coordinate (content) ;
                \node [anchor = north east]  (content) at (content) {
                    \begin{minipage} {4.0cm}
                        \raggedleft \alert{GPUs} can exploit parallelizable nature of matrix operations.
                        % \alert{Computers}, especially \alert{GPUs} can crunch through matrices.
                    \end{minipage}
                }
                ;
            \end{scope}
        \end{scope}

    \end{tikzpicture}


\end{frame}

%% Learning to Ask Questions
\begin{frame} {}

    \begin{tikzpicture} [
            remember picture,
            overlay
        ]

        \tikzframetitleleft{Learning to} {Ask Questions}

        \draw (0, 0)
        coordinate (origin)
        (current page.north east) coordinate (edge)
        ;

        \begin{scope} [
                transform canvas = {
                    yshift = -0.5cm,
                    xshift = -1.25cm,
                }
            ]
            \begin{scope}
                \clip [rotate = -45, rounded corners = 0.7cm]
                (edge) ++(7, 6) rectangle ++(-12, -12)
                ;
                \draw
                (edge) ++(-2, 0.625) coordinate (tmp)
                ;
                \node [anchor = north] (img) at (tmp) {
                    \includegraphics[height = \paperheight]{pics/snetwork.png}
                }
                ;
            \end{scope}

        \end{scope}

        \draw (frametitle.south west) ++(0, -0.3cm) coordinate (content) ;
        \node [anchor = north west]  (content) at (content) {
            \begin{minipage} {6.8cm}
                We haven't addressed how we are going to know the right questions to ask about
                a data yet.
            \end{minipage}
        }
        ;

        \draw (content.south west) ++(0, -0.3cm) coordinate (content) ;
        \node [anchor = north west]  (content) at (content) {
            \begin{minipage} {8.1cm}
                The \alert{coefficients / weights} in our question matrix decides the output
                we get.
            \end{minipage}
        }
        ;

        \draw (content.south west) ++(0, -0.3cm) coordinate (content) ;
        \node [anchor = north west]  (content) at (content) {
            \begin{minipage} {9.3cm}
                And the game of finding these coefficients / weights is famously known under
                the umbrella term \alert{Machine Learning}.
            \end{minipage}
        }
        ;

    \end{tikzpicture}

\end{frame}

%% Weight of the Ever Growing Data
\begin{frame} {}

    \begin{tikzpicture} [
            remember picture,
            overlay
        ]
        \tikzframetitleleft{Weight of} {The Ever Growing Data}

        \begin{scope} [
                rotate = 45,
                transform canvas = {
                    yshift = -1.1cm,
                },
            ]
            \draw ($(current page.center)!0.3!(current page.east)$) coordinate (power);

            \begin{scope}
                \path [name path = mountainHandle, rounded corners = 0.7cm]
                (power) ++(-2.5, -2.5) rectangle ++(5, 5)
                ;
                \clip[use path = mountainHandle] ;
                \begin{scope}
                    \node [anchor = center] (pixel) at (power) {
                        \includegraphics[height = 0.75\paperheight]{pics/mountain.png}
                    }
                    ;
                \end{scope}
            \end{scope}

            \draw (power) ++(5.5, 0) coordinate (power);

            \begin{scope}
                \path [name path = powerHandle, rounded corners = 0.7cm]
                (power) ++(-2.5, -2.5) rectangle ++(5, 5)
                ;
                \clip[use path = powerHandle] ;
                \begin{scope}
                    \node [anchor = center] (pixel) at (power) {
                        \includegraphics[height = 0.75\paperheight]{pics/power.png}
                    }
                    ;
                \end{scope}
            \end{scope}

            \draw (power) ++(-5.5, -5.5) coordinate (power);

            \begin{scope}
                \path [name path = cloudHandle, rounded corners = 0.7cm]
                (power) ++(-2.5, -2.5) rectangle ++(5, 5)
                ;
                \clip[use path = cloudHandle] ;
                \begin{scope}
                    \node [anchor = center] (pixel) at (power) {
                        \includegraphics[height = 0.75\paperheight]{pics/cloud.png}
                    }
                    ;
                \end{scope}
            \end{scope}

            % \draw (power) ++(5.5, 0) coordinate (power);
            %
            % \filldraw [rounded corners = 0.7cm]
            % (power) ++(-2.5, -2.5) rectangle ++(5, 5)
            % ;

        \end{scope}

        \draw (frametitle.south west) ++(0, -0.3cm) coordinate (content) ;
        \node [anchor = north west]  (content) at (content) {
            \begin{minipage} {6.5cm}
                Day by day ML Models are getting more and more capabilities.
                So as the need for more and more powerful hardware to run these models.

                This can lead to several concerns:

                \begin{itemize}
                    \item Increased Carbon Footprint.
                    \item Reliance of Cloud Computing.
                    \item Latency related with Cloud Computing.
                    \item Need for powerful Hardware.
                \end{itemize}

                \alert{TinyML} is a subset of ML that is trying to address these issues.

            \end{minipage}
        }
        ;

    \end{tikzpicture}

\end{frame}

%% An Introduction to Tiny Machine Learning
\begin{frame} {}

    \begin{tikzpicture} [
            remember picture,
            overlay
        ]

        \tikzframetitleright{An Introduction to} {Tiny Machine Learning}

        \draw (frametitle.south east) ++(0, -0.3cm) coordinate (content) ;
        \node [anchor = north east]  (content) at (content) {
            \begin{minipage} {0.5\textwidth}

                TinyML is a subset of machine learning (ML) that focuses on developing and deploying
                ML models on resource-constrained devices such as:

                \begin{itemize}
                    \item Microcontrollers (MCUs)
                    \item System-on-Chip (SoCs)
                    \item FPGAs
                \end{itemize}

            \end{minipage}
        }
        ;

        \def\picWidth{7}
        \def\picGap{0.6cm}

        \draw [
            rounded corners = 0.7cm,
        ]

        ($(current page.center)!0.5!(current page.west)$) coordinate (picCenter)

        % (picCenter) ++(-\picWidth / 2, -\picWidth / 2) rectangle ++(\picWidth, \picWidth)

        (picCenter) ++(-\picWidth / 2, -\picWidth / 2)
        coordinate (picCenterLeft)

        (picCenter) ++(\picWidth / 2, \picWidth / 2)
        coordinate (picCenterRight)

        (current page.north -| picCenterLeft) ++(0, 1) coordinate (picTopLeft)
        (current page.north -| picCenterRight) ++(0, 1) coordinate (picTopRight)

        (current page.south -| picCenterLeft) ++(0, -1) coordinate (picBotLeft)
        (current page.south -| picCenterRight) ++(0, -1) coordinate (picBotRight)

        ;

        \path [
            name path = topPic,
            rounded corners = 0.7cm,
        ]
        (picCenterLeft) -- (picTopLeft) -- (picTopRight) -- (picCenterRight) -- cycle
        ;

        \path [
            name path = botPic,
            rounded corners = 0.7cm,
        ]
        (picCenterLeft) -- (picBotLeft) -- (picBotRight) -- (picCenterRight) -- cycle
        ;

        \begin{scope} [
                transform canvas = {
                    xshift = -0.2cm,
                    yshift = -0.2cm,
                },
            ]
            \begin{scope} [
                    transform canvas = {
                        yshift = \picGap/2,
                    },
                ]

                \clip [use path = topPic] ;

                \draw
                (current page.north west)
                ++(-0.25, 0)
                node [anchor = north west] (img) {
                    \includegraphics[height = 0.85\paperheight]{pics/brain.png}
                }
                ;

            \end{scope}


            \begin{scope} [
                    transform canvas = {
                        yshift = -\picGap/2,
                    },
                ]

                \clip [use path = botPic] ;

                \draw
                (current page.south west)
                node [anchor = south west] (img) {
                    \includegraphics[height = 0.85\paperheight]{pics/chip.png}
                }
                ;

            \end{scope}

        \end{scope}


    \end{tikzpicture}
\end{frame}

%% Why Even Run Machine Learning Models on Microcontrollers?
\begin{frame}

    \begin{tikzpicture} [
            remember picture,
            overlay,
        ]

        \tikzframetitleright{Why Even Run} {Machine Learning Models on Microcontrollers?}

        \filldraw
        ($(current page.center)!0.5!(current page.west)$) ++(3.25cm, -3.5cm) coordinate (center)
        ;

        \subSquare{s1}{center}{padBotCorner} % second
        \subSquare{s2}{center}{padLeftCorner} % third
        \subSquare{s3}{center}{padRightCorner} % first
        \subSquare{s4}{s2-padRightCorner}{padBotCorner} % fourth

        \drawSubSquare{colorLightGreen,}{s1}
        \drawSubSquare{colorLightSageGreen,}{s2}
        \drawSubSquare{colorLightSageGreen,}{s3}
        \drawSubSquare{colorLightGreen,}{s4}

        \node [anchor = center] at (s1-center) {
            \begin{minipage} {\squareSide}
                \centering
                \large Privacy and Security
            \end{minipage}
        }
        ;
        \node [anchor = center] at (s3-center) {
            \begin{minipage} {\squareSide}
                \centering
                \large Low Power \\ Consumption
            \end{minipage}
        }
        ;
        \node [anchor = center] at (s2-center) {
            \begin{minipage} {\squareSide}
                \centering
                \large Improved \\ Latency
            \end{minipage}
        }
        ;
        \node [anchor = center] at (s4-center) {
            \begin{minipage} {\squareSide}
                \centering
                \large Economically \\ Viable
            \end{minipage}
        }
        ;

        \def\picSide{8}

        \begin{scope} [
                transform canvas = {
                    yshift = -0.25cm,
                    xshift = -2.75cm,
                },
                rotate = -45,
            ]
            \clip [
                rounded corners = 0.7cm,
            ]
            (current page.north west)
            ++(0, \picSide/2)
            rectangle ++(\picSide, -\picSide)
            ;

            \draw
            (current page.north west) ++(-5cm, 0)
            node (img) [anchor = north west, inner sep = 0pt] {
                \includegraphics[height = \paperwidth] {pics/bnet.png}
            }
            ;

        \end{scope}

    \end{tikzpicture}

\end{frame}

%% Privacy and Security
\begin{frame} {}


    \begin{tikzpicture} [
            remember picture,
            overlay,
        ]

        \tikzframetitleleft{Privacy} {And Security}

        \def\picSide{7cm}

        \begin{scope}
            \clip[
                rounded corners = 0.7cm,
            ]
            ($(current page.center)!0.5!(current page.east)$)
            ++(-0.75, 0)
            coordinate (center)

            (center) ++(-\picSide/2, -\picSide/2) rectangle ++(\picSide, \picSide)

            ;

            \draw
            (center) ++(-0.75, 0)
            node [anchor = center, inner sep = 0pt] {
                \includegraphics[height = 0.8\paperheight]{pics/military_base.png}
            }
            ;
        \end{scope}


        \draw
        ($(current page.south)!0.5!(current page.south |- frametitle.south)$)
        coordinate (contentCenter)
        (frametitle.west |- contentCenter)
        node [anchor = west] {
            \begin{minipage} {0.425\textwidth}

                \begin{center}
                    ``The most secure connection is no connection at all."
                \end{center}

                \begin{itemize}
                    \item That is the beauty of \alert{Edge Computing}, everything stays
                        locally on the device itself.

                    \item Since the ML Model is running on
                        the device itself, there is no need sent anything over the wire
                        or air.

                    \item This provides the most security and privacy.
                \end{itemize}

            \end{minipage}
        }
        ;

    \end{tikzpicture}

\end{frame}

%% Economically Viable
\begin{frame} {}

    \begin{tikzpicture} [
            remember picture,
            overlay,
        ]

        \tikzframetitleleft{Economically} {Viable}

        \def\picSide{7cm}

        \begin{scope}
            \clip[
                rounded corners = 0.7cm,
            ]
            ($(current page.center)!0.5!(current page.east)$)
            ++(-0.75, 0)
            coordinate (center)

            (center) ++(-\picSide/2, -\picSide/2) rectangle ++(\picSide, \picSide)

            ;
            \draw
            (center)
            node [anchor = center, inner sep = 0pt] {
                \includegraphics[height = 0.8\paperheight]{pics/renew.png}
            }
            ;
        \end{scope}

        \draw
        ($(current page.south)!0.5!(current page.south |- frametitle.south)$)
        coordinate (contentCenter)
        (frametitle.west |- contentCenter)
        ++(0, 0.5)
        node [anchor = west] {
            \begin{minipage} {0.425\textwidth}

                \begin{itemize}
                    \item Modern day Microcontrollers are \alert{extremely cheap}.
                    \item And they are literally used everywhere.
                    % \item So it is highly economical to use these tiny devices than
                    %     to use other \alert{expensive devices}.
                \end{itemize}

                \begin{center}
                    \begin{tabularx} {\textwidth} {
                            >{\centering \arraybackslash}m{0.6\textwidth}
                            >{\centering \arraybackslash}X
                        }
                        \toprule

                        Microcontroller & Price \\ \midrule

                        ESP32 & 550 \\

                        Arduino UNO &  570 \\

                        Raspberry Pi Pico & 450 \\

                        \bottomrule

                    \end{tabularx}
                \end{center}

            \end{minipage}
        }
        ;

    \end{tikzpicture}

\end{frame}

%% Low Power Consumption
\begin{frame} {}

    \begin{tikzpicture} [
            remember picture,
            overlay,
        ]

        \tikzframetitleright{Low Power} {Consumption}

        \def\picSide{7cm}

        \begin{scope}
            \clip[
                rounded corners = 0.7cm,
            ]
            ($(current page.center)!0.5!(current page.west)$)
            ++(0.75, 0)
            coordinate (center)

            (center) ++(-\picSide/2, -\picSide/2) rectangle ++(\picSide, \picSide)

            ;
            \draw
            (center)
            ++(1.25, 0)
            node [anchor = center, inner sep = 0pt] {
                \includegraphics[height = 0.8\paperheight]{pics/battery.png}
            }
            ;
        \end{scope}

        \draw
        ($(current page.south)!0.5!(current page.south |- frametitle.south)$)
        coordinate (contentCenter)
        (frametitle.east |- contentCenter)
        ++(0, 0.5)
        node [anchor = east] {
            \begin{minipage} {0.425\textwidth}

                \begin{itemize}
                    \item Modern day Microcontrollers are \alert{extremely power efficient}.
                    \item While a Raspberry Pi 4 can draw around 1.5A.
                    % \item So it is highly economical to use these tiny devices than
                    %     to use other \alert{expensive devices}.
                \end{itemize}

                \begin{center}
                    \begin{tabularx} {\textwidth} {
                            >{\centering \arraybackslash}X
                            >{\centering \arraybackslash}m{0.4\textwidth}
                        }
                        \toprule

                        Microcontroller & Current Consumption \\ \midrule

                        ESP32 & 20-60 mA \\

                        Arduino UNO & 45-80 mA \\

                        Raspberry Pi Pico & 18-50 mA \\

                        \bottomrule

                    \end{tabularx}
                \end{center}

            \end{minipage}
        }
        ;
    \end{tikzpicture}

\end{frame}

%% Improved Latency
\begin{frame} {}

    \begin{tikzpicture} [
            remember picture,
            overlay,
        ]

        \tikzframetitleright{Improved} {Latency}

        \def\picSide{7cm}

        \begin{scope}
            \clip[
                rounded corners = 0.7cm,
            ]
            ($(current page.center)!0.5!(current page.west)$)
            ++(0.75, 0)
            coordinate (center)

            (center) ++(-\picSide/2, -\picSide/2) rectangle ++(\picSide, \picSide)

            ;
            \draw
            (center)
            ++(-1.5, 0)
            node [anchor = center, inner sep = 0pt] {
                \includegraphics[height = 0.8\paperheight]{pics/latency.png}
            }
            ;
        \end{scope}

        \draw
        ($(current page.south)!0.5!(current page.south |- frametitle.south)$)
        coordinate (contentCenter)
        (frametitle.east |- contentCenter)
        ++(0, 0.5)
        node [anchor = east] {
            \begin{minipage} {0.425\textwidth}

                \begin{itemize}
                    \item No need for cloud processing.

                    \item Everything is \alert{processed locally}.

                    \item So zero network latency.

                    \item Can be deployed to \alert{remote location}.
                \end{itemize}

            \end{minipage}
        }
        ;

    \end{tikzpicture}

\end{frame}

%% The 14 Kilo-Byte "Ok Google" ML Model
\begin{frame} {}

    \begin{tikzpicture} [
            remember picture,
            overlay,
        ]

        \tikzframetitleleft{The 14 Kilo-Byte} {``Ok Google" ML Model}

        \draw (frametitle.south west) ++(0, -0.3cm) coordinate (content) ;
        \node [anchor = north west]  (content) at (content) {
            \begin{minipage} {8.0cm}

                In 2014, Engineers at Google had to implement the ``Ok Google"
                waking feature. But they couldn't use the main CPU for this, as
                it was powered off to conserve battery. So they developed a
                \alert{14 kilo-byte (KB)} neural network that ran on an embedded
                DSPs that had only \alert{tens of kilo-bytes of memory}!

                \vspace{0.5cm}

                \begin{minipage} {6.5cm}

                    These specialised DSPs
                    only used \alert{few milliwatts (mW) of power}.

                \end{minipage}
            \end{minipage}

        }
        ;

        \draw (0, 0)
        coordinate (origin)
        (current page.east) ++(1, 0) coordinate (edge)
        ;

        \begin{scope}

            \draw [
                name path = contentHandle,
            ]
            (content.south) ++(-2.75, -0.3) coordinate (contentMarkEnd)
            (contentMarkEnd) ++(-45:1) coordinate (contentMarkThird)
            (contentMarkThird) ++(7.5, 0) coordinate (contentMarkSecond)
            (contentMarkSecond) ++(45:1) coordinate (contentMarkFirst)
            (contentMarkFirst) ++(45:\markMainOuterRadius) coordinate (contentMarkCenter)
            (contentMarkFirst) -- (contentMarkSecond) -- (contentMarkThird) -- (contentMarkEnd)
            ;

            \draw [use path = contentHandle, very thick] ;
            ;
            \filldraw [fgcol]
            (contentMarkEnd) circle (\markSubRadius)
            ;

            \clip [rotate = 45, rounded corners = 0.7cm]
            (edge) ++(-10, -15) rectangle ++(20, 20)
            ;

            \begin{scope}
                \draw
                (edge) ++(5, 0)
                node [anchor = east] (img) {
                    \includegraphics[height = \paperheight]{pics/phone.png}
                }
                ;

                \filldraw [white!90!fgcol]
                (contentMarkCenter) circle (\markMainInnerRadius)
                ;
                \draw [white!90!fgcol, very thick]
                (contentMarkCenter) circle (\markMainOuterRadius)
                ;
                \draw [use path = contentHandle, white!90!fgcol, very thick] ;
            \end{scope}
        \end{scope}

    \end{tikzpicture}

\end{frame}

% Comparing Modern Day Microcontrollers
\begin{frame} {}

    \begin{tikzpicture} [
            remember picture,
            overlay,
        ]

        \tikzframetitlelongleft{Comparing}{Modern Day Microcontrollers}

        \draw (frametitle.south west) ++(0, -0.1cm) coordinate (content) ;
        \node [anchor = north west]  (content) at (content) {
            \begin{minipage} {\textwidth}

                \centering

                \begin{tabularx} {\textwidth} {
                        >{\centering \arraybackslash}X
                        >{\centering \arraybackslash}X
                        >{\centering \arraybackslash}X
                        >{\centering \arraybackslash}X
                        >{\centering \arraybackslash}X
                    }

                    \toprule

                    \multicolumn{2}{c}{Microcontroller} & Program / Flash Memory & CPU Clock Speed & RAM Available \\
                    \midrule

                    % \includegraphics[height = 2cm]{pics/esp.png} & \multirow{2}{*}{4 MB} & \multirow{2}{*}{240 MHz} & \multirow{2}{*}{520 KB} \\
                    ESP32 & \includegraphics[height = 1.3cm]{pics/esp.png} & 4 MB & 240 MHz & 520 KB \\
                    \midrule

                    % \includegraphics[height = 2cm]{pics/arduino.png} & \multirow{2}{*}{32 KB} & \multirow{2}{*}{16 MHz} & \multirow{2}{*}{2 KB} \\
                    Arduino UNO & \includegraphics[height = 1.3cm]{pics/arduino.png} & 32 KB & 16 MHz & 2 KB \\
                    \midrule

                    % \includegraphics[height = 2cm]{pics/pico.png} & \multirow{2}{*}{2 MB} & \multirow{2}{*}{125 MHz} & \multirow{2}{*}{264 KB} \\
                    Raspberry Pi Pico & \includegraphics[height = 1.3cm]{pics/pico.png} & 2 MB & 125 MHz & 264 KB \\

                    \bottomrule

                \end{tabularx}


            \end{minipage}

        }
        ;

    \end{tikzpicture}

\end{frame}

%% With Great Power Comes Great Responsibility
\begin{frame} {}

    \begin{tikzpicture} [
            remember picture,
            overlay,
        ]

        \tikzframetitleright{With Great Power} {Comes Great Responsibility}

        \draw
        ($(current page.south)!0.5!(current page.south |- frametitle.south)$)
        coordinate (contentCenter)
        (frametitle.east |- contentCenter)
        ++(0, 0.5)
        node [anchor = east] {
            \begin{minipage} {0.575\textwidth}

                Just as TinyML has its advantages, it has its disadvantages.

                \begin{itemize}

                    \item The ML model is \alert{compressed} using various techniques to fit into the
                        \alert{constrains} of the microcontroller.

                    \item The compressed ML model will be \alert{low fidelity} one.

                    \item The low fidelity model \alert{may} or \alert{may not} be applicable for the
                        given task.

                    \item Sometimes we need \alert{cascade} different models having different fidelity.

                \end{itemize}

            \end{minipage}
        }
        ;

        \draw
        ($(current page.center)!0.5!(current page.west)$) ++(-0.25, -4.95cm) coordinate (center)
        ;

        \def\stageGap{0.2}
        \def\stageHeight{2.30}
        \def\baseLen{8}
        \def\stageDec{0.5}

        \foreach \x in {7,6,5} {
            \def\baseLen{\x}
            \filldraw [
                rounded corners = 0.7cm,
            ]
            (center) ++(-\baseLen / 2, 0) coordinate (bl)
            ++(\baseLen, 0) coordinate (br)
            ++(-\stageDec, \stageHeight) coordinate (tr)
            ++({(2 * \stageDec) - \baseLen}, 0) coordinate (tl)

            (center) ++(0, \stageHeight / 2) coordinate (center-\x)
            (bl) -- (br) -- (tr) -- (tl) -- cycle
            ;
            \draw
            (center)
            (center) ++(0, \stageHeight + \stageGap) coordinate (center)
            ;

        }

        \node [white, anchor = center] at (center-7) {
            \begin{minipage} {0.5\textwidth}
                \centering
                Low Fidelity ML Model \\
                (Always running.)
            \end{minipage}
        }
        ;

        \node [white, anchor = center] at (center-6) {
            \begin{minipage} {0.5\textwidth}
                \centering
                Higher Fidelity ML Model \\
                (Triggered from below.)
            \end{minipage}
        }
        ;

        \node [white, anchor = center] at (center-5) {
            \begin{minipage} {0.5\textwidth}
                \centering
                Main CPU \\
                (Triggered from below.)
            \end{minipage}
        }
        ;

        \def\baseLen{4}
        \def\stageDec{1.5}

        \filldraw [
            rounded corners = 0.7cm,
        ]
        (center) ++(-\baseLen / 2, 0) coordinate (bl)
        ++(\baseLen, 0) coordinate (br)
        ++(\stageDec, \stageHeight) coordinate (tr)
        ++({-(2 * \stageDec) - \baseLen}, 0) coordinate (tl)

        (center) ++(0, \stageHeight / 2) coordinate (center-title)
        (bl) -- (br) -- (tr) -- (tl) -- cycle
        ;

        \draw [white]
        (center-title) node [anchor = center] {
            \large Cascading ML Models
        }
        ;

    \end{tikzpicture}

\end{frame}

%% The Essence of Tiny Machine Learning
\begin{frame} {}
    \begin{tikzpicture} [
            remember picture,
            overlay,
        ]

        \tikzframetitlelongright{The Essence of} {Running ML on Microcontrollers}

        \draw
        (current page.center |- frametitle.south)
        node [anchor = north, below = 8pt] {
            \includegraphics[width = 0.95\paperwidth]{tikzpics/endDesignFlow.pdf}
        }
        ;

    \end{tikzpicture}
\end{frame}


\end{document}
