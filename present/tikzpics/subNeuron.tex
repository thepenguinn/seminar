\def\neuronSide{1cm}
\def\neuronGap{0.5cm}

\subtikzpicturedef{subNeuron} {
    bottomCorner,
    rightCorner,
    leftCorner,
    topCorner,
    pBottomCorner,
    pRightCorner,
    pLeftCorner,
    pTopCorner,
    %%
    pBottomRight,
    pBottomLeft,
    pTopRight,
    pTopLeft,
    center,
    origin%
} {
    \draw (#1-start) coordinate (#1-origin);
    \draw
    (#1-origin) coordinate (#1-bottomCorner)
    ++(45:\neuronSide) coordinate (#1-rightCorner)
    ++(135:\neuronSide) coordinate (#1-topCorner)
    ++(-135:\neuronSide) coordinate (#1-leftCorner)

    ($(#1-leftCorner)!0.5!(#1-rightCorner)$) coordinate (#1-center)

    (#1-bottomCorner) ++(0, -\neuronGap) coordinate (#1-pBottomCorner)
    (#1-rightCorner) ++(\neuronGap, 0) coordinate (#1-pRightCorner)
    (#1-leftCorner) ++(-\neuronGap, 0) coordinate (#1-pLeftCorner)
    (#1-topCorner) ++(0, \neuronGap) coordinate (#1-pTopCorner)

    ($(#1-pBottomCorner)!0.5!(#1-pRightCorner)$) coordinate (#1-pBottomRight)
    ($(#1-pRightCorner)!0.5!(#1-pTopCorner)$) coordinate (#1-pTopRight)
    ($(#1-pTopCorner)!0.5!(#1-pLeftCorner)$) coordinate (#1-pTopLeft)
    ($(#1-pLeftCorner)!0.5!(#1-pBottomCorner)$) coordinate (#1-pBottomLeft)

    ;
}

\subtikzpictureactivate{subNeuron}

\newcommand\ovrDrawNeuron[2] {
    \draw [
        #1%
        rounded corners = 0.2cm,
        ultra thick,
    ]

    (#2-bottomCorner) -- (#2-rightCorner) -- (#2-topCorner) -- (#2-leftCorner) -- cycle

    ;
}
