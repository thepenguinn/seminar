\documentclass[fleqn,10pt]{wlscirep}
\usepackage[utf8]{inputenc}
% \usepackage[style = ieee]{biblatex}
\usepackage[T1]{fontenc}
% \usepackage{natbib}
% \usepackage[style=nature]{biblatex}
\title{Pentagonal Patch Antenna with Loaded Open Loop Resonators}

\author[1,*]{Alice Author}
\author[2]{Bob Author}
% \author[1,2,+]{Christine Author}
% \author[2,+]{Derek Author}
\affil[1]{Affiliation, department, city, postcode, country}
\affil[2]{Affiliation, department, city, postcode, country}

% \affil[*]{corresponding.author@email.example}
%
% \affil[+]{these authors contributed equally to this work}

\keywords{Antena, Wifi, Pentagonal}

\begin{abstract}
The microstrip patch antennas are light weight, low profile can be easily fabricated, planar in
configuration and can be printed directly on to circuit boards. The scientific software for
antenna design is High Frequency Structural Simulator (HFSS). By using planar antennas the
cost can be minimized and can easily fabricated. The tool HFSS is very powerful tool in RF
domain especially for designing antennas/filters/Three dimensional structures and all other
active/passive systems.
In this seminar, a pentagonal patch antenna with
loaded open loop resonator (OLR) for improvement of gain and
reduced cross polarization is studied. The standard pentagonal
patch antenna is taken as the parent structure and further
improvement in performance characteristics could be  achieved by
loading the resonator material above the standard pentagonal
patch antenna.  This antenna may
finds applications in long range Wi-Fi communications.
\end{abstract}
\begin{document}

\flushbottom
\maketitle
% * <john.hammersley@gmail.com> 2015-02-09T12:07:31.197Z:
%
%  Click the title above to edit the author information and abstract
%
\thispagestyle{empty}

% \noindent Please note: Abbreviations should be introduced at the first mention in the main text – no abbreviations lists. Suggested structure of main text (not enforced) is provided below.

\section*{Background}
\label{sec:background}
Microstrip patch antennas offer advantages such as light weight, compact size, planar configuration,  ease of fabrication and compatibility with PCB technology \cite{iftissane2011conception}. Various geometries of radiating patch elements such as square, circle, and rectangle have been considered in the previous studies\cite{gupta2013circularly}. However,  microstrip patch antennas generally have narrow bandwidth, low efficiency  due to dielectric and conductor losses and disability to operate at high power levels of waveguide \cite{zhang2016design}\cite{harish20153}\cite{vaishnavi2014simulation}\cite{anand2015analysis}.  The design approach is depicted in Fig. \ref{fig:design}.

\begin{figure} [h]
 \begin{center}

\label{fig:design}
 \includegraphics[width=4.0in]{example-image-a}
 \caption{Design Approach}
 \label{fig:design}
 \end{center}
\end{figure}


\section*{Related Work}
\label{sec:relatedwork}
A pentagonal patch antenna with loaded open loop resonator (OLR) for enhancing the gain and reducing cross-polarization compared to \cite{lafmajani2011miniaturized}\cite{gupta2019design} is proposed. The intended application of the proposed antenna is a technology for providing internet connectivity to fishermen in deep sea using long range Wi-Fi communications \cite{rao2016realizing}\cite{jayakrishnan2018effect}. The specifications of the antenna required for this application are compact size, high gain for providing required connectivity from shore. Directional antennas providing high gain will enable us to achieve high signal to noise ratio such that connectivity in the order of tens of kilometer can be established. Antenna array solutions, on the other hand, although enhances the gain, but will consume large area and hence not suitable for this application. In the proposed antenna configuration, an inset-fed pentagonal microstrip antenna is taken as the parent structure and performance characteristics are studied. The size of the proposed antenna configuration is much less as compared to conventional antennas of similar gain \cite{joshi2011metamaterial}.


% \bibliographystyle{acm}
\bibliography{References}

\end{document}
