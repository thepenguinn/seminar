\documentclass[../../main]{subfiles}

\renewcommand\thesection{\arabic{section}}


\begin{document}

\section{Benefits of TinyML} \label{sec:}

There are several reasons to run TinyML on these tiny devices. Table
\ref{tbl:advantagesTinyML} show some of them.

\begin{center}
    \begin{tabularx} {\textwidth} {
            >{\centering \arraybackslash}X
            >{\centering \arraybackslash}X
        }
        \toprule
        \midrule

        Privacy and Security & Economic Viability \\
        \midrule
        Low Power Consumption & Improved Network Latency \\

        \midrule
        \bottomrule

    \end{tabularx}

    \captionof{table}{Table showing some of the advantages of TinyML.}
    \label{tbl:advantagesTinyML}

\end{center}

We will see each of these in detail in the following sections.

\subsection{Privacy and Security}

The most \emph{secure} connection is \emph{no connection} at all. Since the
ML model is running on the device itself, there is no need to sent anything
over the wire or air. That is the beauty of \emph{edge computing}, everything
stays locally on the device itself. From the privacy and security point of
view this is really desirable.

\subsection{Economic Viability}

Modern day's microcontrollers are \emph{extremely cheap}. And they are
literally used everywhere. So giving them additional capabilities is
much more preferable than replacing them with something that is powerful
but expensive. Refer table \ref{tbl:priceMicrocontrollers} to get
an idea about the market price of some of the popular microcontrollers.

From table \ref{tbl:priceMicrocontrollers} we can see that the average price
of these microcontrollers is around 500 INR. At the same time the cheapest
SBC\footnote{Single Board Computer} costs about 1500 INR. That's just for
the board itself. Inorder to run the device one needs to get other
accessories\footnote{Namely, SD card, power supply, etc.}.

\begin{center}
    \begin{tabularx} {\textwidth} {
            >{\centering \arraybackslash}X
            >{\centering \arraybackslash}X
        }

        \toprule

        Microcontroller & Price (INR) \\ \midrule

        ESP32 & 550 \\

        Arduino UNO &  570 \\

        Raspberry Pi Pico & 450 \\

        \bottomrule

    \end{tabularx}

    \captionof{table}{Table comparing market price of some of the popular microcontrollers.}
    \label{tbl:priceMicrocontrollers}

\end{center}

\subsection{Low Power Consumption}

These tiny devices have advanced much more in the aspect of energy conservation.
They can do much more with much less power consumption. Take a look at table
\ref{tbl:powerMicrocontrollers}, it gives an idea about the current consumption
of some of the microcontrollers.

Comparing to a Raspberry Pi 4, this is really low. Pi 4 consumes about $1.5\si{A}$
of current while running. It is possible to configure these microcontrollers in
their respective power saving mode to bring down the current consumption even less.
ESP32 has a ULP\footnote{Ultra Low Power.} co-processor that can do various tasks
while the main CPU is sleeping for conserving power.

\begin{center}
    \begin{tabularx} {\textwidth} {
            >{\centering \arraybackslash}X
            >{\centering \arraybackslash}X
        }

        \toprule

        Microcontroller & Current Consumption \\ \midrule

        ESP32 & 20-60 mA \\

        Arduino UNO & 45-80 mA \\

        Raspberry Pi Pico & 18-50 mA \\

        \bottomrule

    \end{tabularx}

    \captionof{table}{Table comparing typical current consumption of some of the popular microcontrollers.}
    \label{tbl:powerMicrocontrollers}

\end{center}

\subsection{Improved Network Latency}

This is yet another benefit of \emph{edge computing}. Everything is
processed locally, so the latency related with \emph{cloud computing}
literally become \emph{zero}. This enables these systems to be deployed
to \emph{remote locations} where there have unreliable connections or have
no connection at all.


\end{document}
