\vspace{.25in}
%\begin{flushleft}
%\large{\textbf{~~~Abstract}}\\
%\vspace{0.3in}
%\end{flushleft}
\begin{center}
\large{\textbf{Abstract}}\\
\vspace{0.3in}
\end{center}
\indent

%% Say about TinyML

TinyML is a subset of machine learning (ML) that focuses on developing and deploying
ML models on resource-constrained devices such as Microcontrollers (MCUs), System-on-Chip
(SoCs) and other such devices. Frameworks such as TensorFlow Lite Micro, and Edge Impulse
are used for developing ML models for low powered devices. Techniques used in TinyML
enables these devices to be more efficient at running ML models. Thereby improving
decision-making, reducing cost and power consumption.

%% Say about seminar

This seminar aims to explore what TinyML is, its recent trends and advancements, specifically in its role in the
Agriculture sector. TinyML enables low powered devices such as ESP32 CAM modules, STM32
microcontrollers, and Arduinos to be used in ML applications. These devices can then
perform growth monitoring, plant disease detection, and mirco-climate management in
place of expensive Raspberry Pis and other Mini PCs. Thus, TinyML powered low power
edge devices may find applications in smart agriculture in isolated places and also where
low power consumption is at top priority.
