\documentclass[../../main]{subfiles}

\renewcommand\thesection{\arabic{section}}


\begin{document}

\section{Learning to Ask Questions} \label{sec:}

We have seen that we can extract out any kind of information\footnote{Obviously, only if it
contains the information.} from any kind of data by simply asking it the right questions.
And we can model our questions with matrices. In simple terms, with the \emph{right matrix},
we can extract the desired information from any given data.

Now the question becomes, how we are going find the \emph{right matrix} for a particular problem?
The techniques and methods we use for that is known under the umbrella term \emph{Machine Learning}. Once we
have a pool of data and the desired output\footnote{It may be some class, or a particular feature.},
we can use machine learning techniques\footnote{Like backpropagation and such.} to tweak and tune
the \emph{coefficients} of the question matrix, thereby finding the \emph{right question matrix}.


\end{document}
